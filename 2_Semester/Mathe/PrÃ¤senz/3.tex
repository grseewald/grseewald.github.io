\documentclass[11pt,a4paper]{article}

\usepackage[utf8]{inputenc}
\usepackage[ngerman]{babel}
\usepackage{amsmath}
\usepackage{enumerate}
\usepackage{fancyhdr}
\usepackage{xfrac}
\usepackage{amssymb}
\usepackage{mathtools}
\usepackage{marvosym}
\pagestyle{fancy}

\begin{document}
	Punktweise und Gleichmäßige Konvergenz\\
		Es sei $f(n):D\rightarrow\mathbb{R}$ eine Folge von Funktionen, dann heißt $(f_n)n_{n\in \mathbb{N}}$ punktweise Konvergent falls die Folge $(f_n(x))_n$ für alle 	$x\in D$ konvergiert. Die Funktionen-Folge $(f_n)_{n\in  \mathbb{N}}$ heißt gleichmäßig Konvergent gegen $f$ falls\\
		 $\forall x \forall \epsilon>0 \exists N_0\in \mathbb{N} \forall n \geq N_0 \forall x \in D |f_n(x)-f(x)|<\epsilon$ oder äquivalent\\
		 $\forall \epsilon >0 \exists N_0\in \mathbb{N}\forall n\geq N_0 \sup(|f_n(x)-f(x)|)<\epsilon$\\
		 Satz: Ist $(f_n)_n$ eine gleichmäßig Konvergente Folge stetiger Funktionen, dann ist ihr Grenzwert auch stetig.\\
		 Bsp: Sei $f_n:[0,1]\rightarrow \mathbb{R}\qquad f_n(x)=x^n$ hat Punktmäßig Grenzwerte.\\
		 \begin{equation} 
   			f(x) = 
   			\begin{cases} 
     			0 \text{ für alle }x\in [0,1)\\
     			1 \text{ für} x=1
   			\end{cases} 
		\end{equation} 
		Der Grenzwert ist nicht stetig also Konvergenz nicht gleichmäßig.\\
		Warum wichtig für Infos?\\
		Oft ist idealer Weg gesucht, z.B Flugbahn, Fräsbahn. Diese wird durch Funktion $\gamma:[0,1]\rightarrow \mathbb{R}^3$ beschrieben. Nur passen allgemeine Funktionen nicht in den Computer rein.\\
		Man behilft sich durch diskretisierung $\gamma_n$ der Funktion $\gamma_n[0,\frac{1}{h},\frac{2}{h},\frac{3}{h},...,1]\rightarrow\mathbb{R}^3$\\
		Dieses $\gamma_n$ soll für jede Folge $h_n\rightarrow 0$ gleichmäßig gegen $\gamma$ konvergieren.\\
		\section*{Aufgabe 1} 
			Zeigen sie die Funktionen-Folge $f_n:\mathbb{R}\rightarrow\mathbb{R} mit$
			\begin{equation}
				f_n(x):=
				\begin{cases}
					0 \text{ für } x\in \mathbb{R}\text{\textbackslash}\mathbb{Q} \\
					\frac{1}{n} \text{ für } x\in \mathbb{Q}
				\end{cases}
			\end{equation}
			konvergiert gleichmäßig.
		\section*{Aufgabe 2}
			Es sei $f_n:D\rightarrow\mathbb{R}$ mit $f_n(x) = \frac{1}{n}x$ eine Funktionen-Folge.\\
			Zeigen sie:\\
			\begin{enumerate}[a)]
				\item Ist $D=[a,b]$ mit $a,b\in \mathbb{R}, a\in b$ beliebig so konvergiert $(f_n)_n$ gleichmäßig auf $D$
				\item Ist $D=\mathbb{R}$ so konvergiert $(f_n)_n$ zwar punktweise aber nicht gleichmäßig.
			\end{enumerate}
		\section*{Lösung zu Aufgabe 1}
			Es sei $\epsilon >0$ beliebig (aber fest).\\
			 Wähle $N_0\in \mathbb{N}$, so dass $N_0>\frac{1}{\epsilon}$ dann gilt für alle $n\geq N_0$ und alle $x\in D$ $|f_n(x)-0|\leq\frac{1}{n}\leq\frac{1}{N_0}<\epsilon$.\\
			Damit ist gleichmäßige Konvergenz von $f_n$ gegen die Null-Funktion gezeigt.\\
		\section*{Lösung 2a)}
			Es sei $\epsilon > 0$ (beliebig aber fest).\\
			Wähle $N_0\in \mathbb{N}$, so dass\\
			$N_0>\frac{1}{\epsilon}\cdot max{|a|,|b|}$\\
			Dann gilt für alle $n\geq N_0$ und alle $x\in D$, dass\\
			$|f_n(x)-0|=|\frac{1}{n}x|\leq\frac{1}{n}|x|\leq\frac{1}{n}max{|a|,|b|}\leq\frac{1}{N_0}\cdot max{|a|,|b|}<\epsilon$
		\section*{Lösung 2b)}
		Es sei $x\in \mathbb{R}$ beliebig aber fest.\\
		Es sei $\epsilon >0$ beliebig aber fest.\\
		Dann wähle $N_0\in\mathbb{N} N_0>\frac{1}{\epsilon}\cdot|x|$\\
		Dann gilt für alle $n\geq N_0$, dass $|f_n(x)-0|=\frac{1}{n}\cdot|x|\leq\frac{1}{N_0}|x|<\epsilon$.\\
		Zur gleichmäßigen Konvergenz.\\
		Angenommen $(f_n)$ würde gleichmäßig konvergieren, dann da $f_n(x)\stackrel{P}{\rightarrow}0$ müsste der gleichmäßige Grenzwert auch die Nullfunktion sein.\\
		Wähle $\epsilon=\frac{1}{2}$ und setze $x_n=n$, dann gilt $|f_n(x_n)-0|=1>\frac{1}{2}=\epsilon$.\\
		Also:\\
		$\exists\epsilon >0\forall N_0\in\mathbb{N} \exists n\geq N_0 \exists X |f_n(x)-0|\geq \epsilon$
\end{document}