\documentclass[11pt,a4paper]{article}

\usepackage[utf8]{inputenc}
\usepackage[ngerman]{babel}

\usepackage{amsmath}
\usepackage{amssymb}
\usepackage{mathtools}
\usepackage{xfrac}
\usepackage{marvosym}
\usepackage{wasysym}

\usepackage{enumerate}


\DeclarePairedDelimiter\ceil{\lceil}{\rceil}
\DeclarePairedDelimiter\floor{\lfloor}{\rfloor}

\DeclarePairedDelimiter\abs{\lvert}{\rvert}%
\DeclarePairedDelimiter\norm{\lVert}{\rVert}%

% Swap the definition of \abs* and \norm*, so that \abs
% and \norm resizes the size of the brackets, and the
% starred version does not.
\makeatletter
\let\oldabs\abs
\def\abs{\@ifstar{\oldabs}{\oldabs*}}
%
\let\oldnorm\norm
\def\norm{\@ifstar{\oldnorm}{\oldnorm*}}
\makeatother

\begin{document}
  \section*{Zum Beweisen}
    Seien $A$ und $B$ zwei Aussagen gegeben. Wir wollwn $A \Rightarrow B$ zeigen. Dazu gibt es drei Beweis Arten.\\
    \subsection*{Direkter Beweis}
      $A \Rightarrow ... \Rightarrow ... \Rightarrow B$
    \subsection*{Kontraposition}
      $\neg B \Rightarrow ... \Rightarrow ... \Rightarrow \neg A$
    \subsection*{Widerspruchs Beweis}
      $A \wedge \neg B \Rightarrow ... \Rightarrow ... \Rightarrow \text{\Lightning}$
    \subsection{Bsp}
      Satz aus der Vorlesung:\\
        Sind $f,g$ differenzierbare Funktionen, dann sind auch $f+g,f-g,f \cdot g, f \circ g$ differenzierbar, sowie $\frac{f}{g}$ falls $g(x)>0 \forall x$\\
      Aufgabe:\\
        Sei $f,g:(0,\infty)\to(0,\infty)$.\\
        Zeige $f$ und $g$ differenzierbar ist äquivalent zu $f+g^2$ und $f-g^2$ differenzierbar.\\
      Beweis: Wir zeigen zwei Richtungen:\\
        \glqq$\Rightarrow$\grqq Sind $f$ und $g$ differenzierbar, so ist nach dem Satz aus der Vorlesung auch $g^2$ sowie sowie $f+g^2$ und $f-g^2$ differenzierbar \glqq$\Leftarrow$\grqq Sind nun $f+g^2$ und $f-g^2$ differenzierbar, so sind auch $\frac{1}{2}\left(f+g^2 \right)+\frac{1}{2}\left(f-g^2\right)=f$ und $\frac{1}{2}\left( f+g^2 \right)- \frac{1}{2}\left( f-g^2 \right)=g^2$ differenzierbar.\\
        Da nach der Vorlesung die Wurzelfunktion differenzierbar und $g\left( x \right)>0$ ist auch $\sqrt{g\left( x\right)} = g\left( x \right)$ differenzierbar.
  \section*{Stetig differenzierbare Funktion}
    $C^n\left( \left( a,b \right) \right)=\left\lbrace f:\left(a,b\right) \to \mathbb{R} | \text{$f$ ist $n$ mal differenzierbar und $f^{\left( i\right)}$ ist stetig in $\left( a,b \right)$}\right\rbrace$\\
    $C^n\left( \left[ a,b \right]\right)=\left\lbrace f:\left(a,b\right) \to \mathbb{R} | \text{$f$ ist $n$ mal stetig differenzierbar in $\left( a,b \right)$ und $f^{ \left( i\right) }$, $i=0,...,n$ ist stetig in $\left[ a,b \right] $ fortsetzbar}\right\rbrace$
    \newpage
  \section'{Aufgabe mit dem Taylor}
    \begin{enumerate}[a)]
      \item Zeige:\\
        $u'(x)=\frac{u(x+h)-u(x)}{h}+o(h)$\\
        d.h.: Es gib $r:\mathbb{R}\to\mathbb{R}$ mit\\
        $u'(x)=\frac{u(x+h)-u(x)}{h}+r(h)$\\
        wobei $\frac{r(h)}{h}\to 0$ für $h\to 0$
      \item Zeige $u'(x)=\frac{u(x+h)-u(x-h)}{2h}+o(h^2)$
      \item Zeige $-u''(x)=\frac{2u(x)-u(x+h)-u(x-h)}{h^2}+o(h^2)$
    \end{enumerate}
    \subsection{Lösung}
      \begin{enumerate}[a)]
        \item Mit dem Satz von Taylor folgt:\\
          $u(x+h)=u(x)+u'(x)h+o(h) \Leftrightarrow u'(x)=\frac{u(x+h)-u(x)}{h}+o(h)$
        \item
        \item Mit dem Satz von Taylor folgt:\\
        $u(x+h)=u(x)+u'(x)h+\frac{1}{2}u''(x)h^2$\\
        $u(x-h)=u(x)-u'(x)h+\frac{1}{2}u''(x)(-h)^2+o(h^2)=u(x)-u'(x)h+\frac{1}{2}u''(x)h^2+o(h^2)$\\
        Subtrahieren liefert:\\
        $u(x+h)-u(x-h)=0+2u'(x)h+0+o(h^2)\Leftrightarrow b$\\
        Addieren liefert c)\\
        $\frac{2r(h)}{h^2} \to 0$\\
        $o(h^2)\pm o(h^2)=o(h^2)$
      \end{enumerate}
\end{document}
