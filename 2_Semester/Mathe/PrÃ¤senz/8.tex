\documentclass[11pt,a4paper]{article}

\usepackage[utf8]{inputenc}
\usepackage[ngerman]{babel}

\usepackage{amsmath}
\usepackage{amssymb}
\usepackage{mathtools}
\usepackage{xfrac}
\usepackage{marvosym}

\usepackage{enumerate}

\usepackage{fancyhdr}

\usepackage{soul}

\usepackage{array}   % for \newcolumntype macro
\newcolumntype{L}{>{$}l<{$}} % math-mode version of "l" column type
\newcommand{\Aufgabe}[1]{\section*{Aufgabe #1}}
\newcommand{\lb}[0]{\ensuremath{\left\lbrace}}
\newcommand{\rb}[0]{\ensuremath{\right\rbrace}}


\pagestyle{fancy}

\fancyhf[HR]{Lukas Vormwald, Noah Mehling, Gregor Seewald}
\fancyhf[HL]{Übung:Dienstag 12:00}

\title{Präsenz 8}
\author{Gregor Seewald}
\date{\today}
\DeclarePairedDelimiter\ceil{\lceil}{\rceil}
\DeclarePairedDelimiter\floor{\lfloor}{\rfloor}

\DeclarePairedDelimiter\abs{\lvert}{\rvert}%
\DeclarePairedDelimiter\norm{\lVert}{\rVert}%

% Swap the definition of \abs* and \norm*, so that \abs
% and \norm resizes the size of the brackets, and the
% starred version does not.
\makeatletter
\let\oldabs\abs
\def\abs{\@ifstar{\oldabs}{\oldabs*}}
%
\let\oldnorm\norm
\def\norm{\@ifstar{\oldnorm}{\oldnorm*}}
\makeatother

\begin{document}

\maketitle

  Aufgabe: Wir haben eine Urne mit 3 schwarzen(s), 5 weißen(w) und 7 grünen/grauen(g) Kugeln.\\
  \begin{enumerate}[a)]
    \item Wir ziehen einmal:
      \begin{enumerate}[i)]
        \item Was ist der Wahrscheinlichkeitsraum $\Omega_1$ und was ist eine sinnvolle Annahme für ein Wahrscheinlichkeitsmaß $P_1$?[Was ist $P(\left\lbrace w\right\rbrace),P(\left\lbrace s\right\rbrace),P(\left\lbrace g\right\rbrace)$]
        \item Wie hoch ist die Wahrscheinlichkeit eine Schwarze oder eine Weiße Kugel zu ziehen?(Mit oben definiertem W-Maß.)
        \item Modelliere das Problem aus $a)ii)$ mit Hilfe einer Zufallsvariable $\st{x}:\left\lbrace w,s,g\right\rbrace\to\mathbb{R}$
      \end{enumerate}
    \item Nun ziehen wir zweimal mit zurücklegen
      \begin{enumerate}[i)]
        \item Was ist Wahrscheinlichkeitsraum $\Omega_2$?\\
        Was ist sinnvolles Wahrscheinlichkeitsmaß $P_2$(basierend auf $P_1$?\\
        \item Wie hoch ist die Wahrscheinlichkeitsma
          \begin{enumerate}[1)]
            \item zwei weiße Kugeln zu ziehen
            \item eine weiße und eine schwarze zu ziehen
            \item keine weiße zu ziehen
            \item Sind die Ergebnisse 1&2 stochastisch unabhängig
          \end{enumerate}
        \item Wie hoch ist die Wahrscheinlichkeit
          \begin{enumerate}[1)]
            \item als erstes eine Grüne zu ziehen
            \item als zweites eine WEiße zu ziehen
            \item Sind die Ereignisse 1&2 stochastisch unabhängig
          \end{enumerate}
      \end{enumerate}
      \item Wo ist das Problem unendliches ziehen zu modellieren?
  \end{enumerate}

  \paragraph{Lösung:}
  \begin{enumerate}[a)]
    \item
    \begin{enumerate}[i)]
      \item Der Wahrscheinlichkeitsraum $\Omega_1=\left\lbrace s,w,g \right\rbrace$\\
        Um das Wahrscheinlichkeitsmaß zu definieren definiere die Zähldichte $f:\left\lbrace s,w,g \right\rbrace \to \mathbb{R}$ als $f(w)=\frac{3}{15},f(s)=\frac{5}{15},f(g)=\frac{7}{15}$ und definiere ein $A\in\left\lbrace w,s,g \right\rbrace$, $P:2^\Omega\to[0,1]$\\
        $P_1(A)=\sum\limits_{z\in A} f(z)$
      \item
      \begin{align*}
        P_1(\left\lbrace w,s \right\rbrace)&=P_1(\lb w\rb)+P_1(\lb s \rb)\\
        &=f(w)+f(s)\\
        &=\frac{3}{15}+\frac{5}{15}\\
        &=\frac{8}{15}
      \end{align*}
    \end{enumerate}
    \item
    \begin{enumerate}[i)]
      \item Es ist \\
      \begin{align*}
        \Omega=\lb &s,w,g\rb^2\\
        =\lb &(w,w),(w,s),(w,g),\\
        &(s,w),(s,s),(s,g),\\
        &(g,w),(g,s),(g,g) \rb
      \end{align*}
      Ein sinnvolles Wahrscheinlichkeitsmaß ist gegeben durch:\\
      $P(B)=\sum\limits_{(Z_1,Z_2)\in B} P_1(\lb Z_1\rb)\cdot P_1(\lb Z_2\rb)$
      \item \begin{enumerate}[1)]
        \item $P_2(\lb (w,w)\rb)=P_1(\lb w \rb)\cdot P_1(\lb w\rb)=\left(\frac{5}{15}\right)^2=\frac{1}{9}$
        \item $P_2(\lb (w,s),(s,w)\rb)=P_2(\lb (w,s)\rb)+P_2(\lb (s,w)\rb) = 2 \cdot P_1(\lb s\rb) \cdot P_1(\lb w\rb)= \frac{2}{15}$
        \item $P_2(\lb (s,s),(s,g),(g,s),(g,g)\rb)=\frac{4}{9}$
        \item Keine zwei weißen\\
        $P_2(\lb (w,w)\rb^c)=1-P(\lb (w,w)\rb)$ 
      \end{enumerate}
    \end{enumerate}
  \end{enumerate}


\end{document}
