\documentclass[11pt,a4paper]{article}

\usepackage[utf8]{inputenc}
\usepackage[ngerman]{babel}

\usepackage{amsmath}
\usepackage{amssymb}
\usepackage{mathtools}
\usepackage{xfrac}
\usepackage{marvosym}

\usepackage{enumerate}

\usepackage{fancyhdr}

\usepackage{array}   % for \newcolumntype macro
\newcolumntype{L}{>{$}l<{$}} % math-mode version of "l" column type
\newcommand{\Aufgabe}[1]{\section*{Aufgabe #1}}
\newcommand{\meth}[1]{\ensuremath{\mathbb{#1}}}


\pagestyle{fancy}

\fancyhf[HR]{Lukas Vormwald, Noah Mehling, Gregor Seewald}
\fancyhf[HL]{Übung:Dienstag 12:00}

\title{Mathematik für Informatiker 2\\Übungsblatt 8}
\author{Lukas Vormwald \and Noah Mehling \and Gregor Seewald}
\date{Übung 5:Dienstag 12:00}
\DeclarePairedDelimiter\ceil{\lceil}{\rceil}
\DeclarePairedDelimiter\floor{\lfloor}{\rfloor}

\DeclarePairedDelimiter\abs{\lvert}{\rvert}%
\DeclarePairedDelimiter\norm{\lVert}{\rVert}%

% Swap the definition of \abs* and \norm*, so that \abs
% and \norm resizes the size of the brackets, and the
% starred version does not.
\makeatletter
\let\oldabs\abs
\def\abs{\@ifstar{\oldabs}{\oldabs*}}
%
\let\oldnorm\norm
\def\norm{\@ifstar{\oldnorm}{\oldnorm*}}
\makeatother

\begin{document}

  Es sei $\mathbb{K}$ ein $\mathbb{K}=\mathbb{R}$\\
  Einfaches Nachrechnen zeigt die Polynomfunktion mit Grad $\leq n$\\
  $p:\mathbb{R}\to\mathbb{R}$ mit $p(x)=a_nx^n+...+a_1x+a_0\qquad a_i\in\mathbb{R},i=1...n$\\
  bilden einen Vektorraum.\\
  Das additiv neutrale Element \underline{ist das Nullpolynom/die Nullfunktion}\\
  \paragraph{Aufgabe:}
    Zeige, dass $\left\lbrace \tilde{p_0}=1,\tilde{p_1}=x,\tilde{p_2}=x^2\right\rbrace$\\
    und\\
    $\left\lbrace p_0(x)=(x-1)(x+1),p_1(x)=x(x+1),p_2(x)=x(x-1)\right\rbrace$\\
    eine Basis von $\mathbb{P}_2 bilden$\\
  \paragraph{Lösung zur ersten Menge:}
    Lineare Unabhängigkeit\\
    Die Polynomefunktion $\tilde{p_0},\tilde{p_1},\tilde{p_2}$ sind linear unabhängig, da mit $\lambda_0,\lambda_1,\lambda_2\in\mathbb{R}$ und\\
    $0=\lambda_0 p_0(x)+\lambda_1 p_1(x)+\lambda_2 p_2(x)\\
    =\lambda_0 \cdot 1 +\lambda_1 \cdot x +\lambda_2 \cdot x^2$\\
    Dies ist nur das Nullpolynom falls $\lambda_0=\lambda_1=\lambda_2=0$\\
    Bzw. die Nullfunktion\\
    Erzeugenden-System dem $\meth{P}_2$ ist gerade als $\meth{P}_2=\left\lbrace a_2x^2+a_1x+a0, a_2,a_1,a_0\in\meth{R}\right\rbrace$ definiert. Also $dim\meth{P}_2=3$\\
  \paragraph{Lösung zur zweiten Menge:}
    s




\end{document}
