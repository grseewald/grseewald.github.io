\documentclass[11pt,a4paper]{article}

\usepackage[utf8]{inputenc}
\usepackage[ngerman]{babel}

\usepackage{amsmath}
\usepackage{amssymb}
\usepackage{mathtools}
\usepackage{xfrac}

\usepackage{enumerate}

\DeclarePairedDelimiter\ceil{\lceil}{\rceil}
\DeclarePairedDelimiter\floor{\lfloor}{\rfloor}

\DeclarePairedDelimiter\abs{\lvert}{\rvert}%
\DeclarePairedDelimiter\norm{\lVert}{\rVert}%

% Swap the definition of \abs* and \norm*, so that \abs
% and \norm resizes the size of the brackets, and the
% starred version does not.
\makeatletter
\let\oldabs\abs
\def\abs{\@ifstar{\oldabs}{\oldabs*}}
%
\let\oldnorm\norm
\def\norm{\@ifstar{\oldnorm}{\oldnorm*}}
\makeatother

\begin{document}
  $f(x) = \frac{1}{2} \left( x + \abs{x}\right) \sqrt{\abs{x}}$\
  Für $x>0$ gilt: $f\left( x\right) = \frac{1}{2} \left( x + \abs{x}\right) \sqrt{\abs{x}} = \frac{1}{2} \left( x + x \right) \sqrt{x} = x^{\frac{3}{2}}$\\
  	$\Rightarrow f$ ist differenzierbar.......
	\section*{Aufgabe 2}
		Zunächst gilt $\abs{f\left( 0 \right)} \leq 0^2 = 0$, also $f\left( 0 \right)=0$.\\
		Wegen $0 \leq \abs{ \frac{f\left( x\right) - f\left( 0\right)}{x-0}} = \frac{\abs{f\left( x\right)}}{x}$
	\section*{Aufgabe 3}
  		Die Funktion $f:\mathbb{R} \to \mathbb{R}$, $x \to \ln\left( 1 + e^x\right)$ ist differenzierbar mit $f'\left( x \right) = \frac{e^x}{1 + e^x}$. Für festes $x,y \in \mathbb{R}, x \neq y$ gibt es nach dem Mittelwertsatz ein $\xi \in \mathbb{R}$ zwischen $x$ und $y$ mit\\
  		$\frac{f\left( x\right) - f\left( y\right)}{x - y} = f'\left( \xi \right)$. Also folgt $\abs{ \ln \frac{1 + e^x}{1 + e^y}} = \abs{ f\left( x\right) - f\left( y\right)} = \abs{ f'\left( \xi \right)}\abs{x - y} = \frac{e^{\xi}}{1 + e^\xi}\abs{x-y}\leq\abs{x-y}$
  	\section*{Aufgabe 4}
  		Für jedes $x \in \left[ -1, 1\right]$ gibt es nach dem Satz von Taylor ein $\xi \in \left[ -1,1 \right]$ mit $f\left( x \right) = f\left( 0 \right) + f'\left( x \right)\left( x - 0 \right) + \frac{f''\left( \xi \right)}{2}\left( x - 0 \right)^2$\\
  		Da $f''$ stetig auf dem abgeschlossenen und beschränkten Intervall $\left[ -1,1 \right]$ ist, ist auch $f''$ dort beschränkt (Satz von Weierstraß); insbesondere gibt es ein $c>0$ mit $\abs{\frac{f''\left( \xi \right)}{2}} \leq c \forall \xi \in \left[ -1,1 \right]$\\
  		$\Rightarrow \abs{f\left( x \right)} = \abs{\frac{f''\left( \xi \right)}{2}} x^2 \leq cx^2 \forall x \in \left[ -1,1\right]$
  	\section*{Aufgabe 5}
  		Die Funktion $f\left( x \right) = \ln \left( 1 + x \right)$ ist beliebig oft differenzierbar für $x>-1$ mit $f'\left( x \right) = \frac{1}{1 + x}, f''\left( x \right) = - \frac{1}{\left( 1+x^2\right)}, f'''\left( x \right) = \frac{2}{\left( 1 + x\right)^3}$. Nach dem Satz von Taylor gibt es zu jedem $n \in \mathbb{N}$ und $x=\frac{1}{10}$ ein $\xi \in \left( 0,\frac{1}{10}\right)$ mit $f\left( \frac{1}{10}\right)=T_n\left(\frac{1}{10},0\right)+R_n\left(\frac{1}{10},0\right)$, wobei $R_n\left( \frac{1}{10},0\right)=\frac{f^{\left( n+1 \right)}\left( \xi \right)}{\left( n+1\right)!}\left( \frac{1}{10}-0\right)^{n+1}$ bezeichne.\\
  		$\Rightarrow \abs{T_n (\frac{1}{10},0)-\ln(1,1)}=\abs{R_n(\frac{1}{10},0)}$. Für $n=2$ erhält man $\abs{R_2(\frac{1}{10},0)}=\frac{\abs{f'''(\xi)}}{6\cdot 10^3}= \frac{2}{(1+\xi)^3\cdot 6 \cdot 10^3}=\frac{1}{3(1+\xi)^3}\cdot 0,001 \leq \frac{1}{3}\cdot 0,001 \leq 0,001$\\
  		Für $q:=T_2(\frac{1}{10},0)=f(0)+f'(0)$.......
  	\section*{Aufgabe 6}
  		Zunächst gilt nach dem Satz von Taylor $q(x)=\sum\limits_{j=0}^p a_j(x-x_0)^j$ für $a_j \in \mathbb{R}, x\in B_\epsilon (x_0)$. Wir zeigen durch induktion nach $k$, dass $a_k=0 \forall k\in \left\lbrace 0,...,p\left\rbrace$.
\end{document}
