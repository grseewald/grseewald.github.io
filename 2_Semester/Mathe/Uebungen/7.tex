\documentclass[11pt,a4paper]{article}

\usepackage[utf8]{inputenc}
\usepackage[ngerman]{babel}

\usepackage{amsmath}
\usepackage{amssymb}
\usepackage{mathtools}
\usepackage{xfrac}
\usepackage{marvosym}

\usepackage{enumerate}

\usepackage{fancyhdr}

\usepackage{array}   % for \newcolumntype macro
\newcolumntype{L}{>{$}l<{$}} % math-mode version of "l" column type


\pagestyle{fancy}

\fancyhf[HR]{Lukas Vormwald, Noah Mehling, Gregor Seewald}
\fancyhf[HL]{Übung:Dienstag 12:00}

\title{Mathematik für Informatiker 2\\Übungsblatt 7}
\author{Lukas Vormwald \and Noah Mehling \and Gregor Seewald}
\date{Übung 5:Dienstag 12:00}
\DeclarePairedDelimiter\ceil{\lceil}{\rceil}
\DeclarePairedDelimiter\floor{\lfloor}{\rfloor}

\DeclarePairedDelimiter\abs{\lvert}{\rvert}%
\DeclarePairedDelimiter\norm{\lVert}{\rVert}%

% Swap the definition of \abs* and \norm*, so that \abs
% and \norm resizes the size of the brackets, and the
% starred version does not.
\makeatletter
\let\oldabs\abs
\def\abs{\@ifstar{\oldabs}{\oldabs*}}
%
\let\oldnorm\norm
\def\norm{\@ifstar{\oldnorm}{\oldnorm*}}
\makeatother

\begin{document}

  \section*{Aufgabe 1}

    \begin{enumerate}[a)]
      \item
    \end{enumerate}

  \section*{Aufgabe 2}

    $\frac{1}{1-z-z^2-z^3}$\\
    Nullstellen von Nennerpolynom:\\
    $z=1$\\
    $(z^3-z^2-z+1) : (z-1)=z^2-1$\\
    $\Rightarrow z^2=1 \Rightarrow z_2=1,z_3=-1$\\
    $\Rightarrow \frac{1}{\left(z-1\right)^2\cdot\left(z+1\right)}=\frac{A}{\left(z+1\right)}+\frac{B}{\left(z-1\right)}+\frac{C}{\left(z-1\right)^2}$\\
    $\Rightarrow 1=\frac{A\cdot \left(z-1\right)^2-\left(z+1\right)}{\left(z+1\right)}+\frac{B\cdot \left(z-1\right)^2\cdot \left(z+1\right)}{\left(z-1\right)}+\frac{C\cdot \left(z-1\right)^2\cdot\left(z+1\right)}{\left(z-1\right)^2}$\\
    $=A\cdot \left(z-1\right)^2+B\cdot\left(z-1\right)\left(z+1\right)+C\cdot\left(z+1\right)$\\
    $\Rightarrow z=1 \Rightarrow 1=2C \Rightarrow C=0,5$\\
    $z=-1 \Rightarrow 1=4A \Rightarrow A=0,25$\\
    $\Rightarrow$ z beliebig, z.B $0$\\
    \begin{align*}
      \Rightarrow 1 =& 0,25 \cdot 1 + B \cdot \left( -1 \right) \cdot 1 + 0,5 \cdot 1\\
      1 =& 0,25 - B + 0,5\\
      B =& 0,75 - 1 + 0,5 &\Rightarrow B = -0,25
    \end{align*}
    $\Rightarrow \frac{1}{1-z-z^2+z^3} = \frac{0,25}{\left(z+1\right)} - \frac{0,25}{\left(z-1\right)}+ \frac{0,5}{\left(z-1\right)^2}$

  \section*{Aufgabe 3}

  \section*{Aufgabe 4}

    $\frac{2x^2 + 7x + 5}{\left( x^2 + x + 2\right)\left( x^2 +1 \right)}$\\
    Nullstellen:\\
    $\left(x^2+1\right)\Rightarrow$ keine\\
    $x^2+x+2 \Rightarrow x_{1,2} = \frac{-1\pm\sqrt{1^2-4\cdot1\cdot2}}{2-1}$\qquad\Lightning keine Nullstelle\\
    $\Rightarrow \frac{Ax+B}{4x^2+x+2} + \frac{Cx+D}{x^2+1}$\\
    $\Rightarrow \frac{2x^2+7x+5}{\left(x^2+x+2\right)\left(x^2+1\right)} = \frac{Ax+B}{x^2+x+2} + \frac{Cx+D}{x^2+1}$
    \newpage
    \begin{align*}
      \frac{2x^2+7x+5}{\left(x^2+x+2\right)\left(x^2+1\right)} =& \frac{\left(Ax+B\right)\left(x^2+1\right)}{\left(x^2+x+2\right)\left(x^2+1\right)} + \frac{\left(Cx+D\right)\left(x^2+x+2\right)}{\left(x^2+x+2\right)\left(x^2+1\right)}\\
      =& \frac{\left(Ax+B\right)\left(x^2+1\right) + \left(Cx + D\right)\left(x^2+x+2\right)}{\left(x^2+x+2\right)\left(x^2+1\right)}\\
      =& \frac{Ax^3+Ax+Bx^2+B+Cx^3+Cx^2+2Cx+Dx^2+Dx+D}{\left(x^2+x+2\right)\left(x^2+1\right)}\\
    \frac{2x^2+7x+5}{\left(x^2+x+2\right)\left(x^2+1\right)} =& \frac{x^3\left(A+C\right)+x^2\left(B+C+D\right)+x\left(A+2C+D\right)+\left(B+D\right)}{\left(x^2+x+2\right)\left(x^2+1\right)}
    \end{align*}
    \begin{align}
      \Rightarrow A+C=&0 & \Rightarrow A=&C \label{eq:1}\\
      B+C+D=&2 \label{eq:2}\\
      A+2C+D=&7 \label{eq:3}\\
      B+D=&5 \label{eq:4}
    \end{align}
    \begin{align}
      \Rightarrow \eqref{eq:1} \text{ in } \eqref{eq:3} \Rightarrow A+2A+D=7 \Rightarrow 3A+D=7 \Rightarrow D=7-3A \label{eq:3*}\\
      \Rightarrow \eqref{eq:1} \& \eqref{eq:3*} \text{ in } \eqref{eq:2} \Rightarrow B+A+7-3A=2 \Rightarrow B-2A=-5 \Rightarrow B=2A-5 \label{eq:2*}\\
      \Rightarrow \eqref{eq:2*} \& \eqref{eq:3*} \text{ in } \eqref{eq:4} \Rightarrow 2A-5+7-3A=5 \Rightarrow 2A-3A=3 \Rightarrow -A=3 \Rightarrow A=-3
    \end{align}
    \begin{align*}
      B&=2\cdot \left(-3\right)-5 &= 11\\
      C&=A&=-3\\
      D&=7-3\cdot\left(-3\right) &=7+9 &=16
    \end{align*}
    \begin{equation*}
      \Rightarrow \frac{2x^2+7x+5}{\left(x^2+x+2\right)\left(x^2+1\right)}&\frac{-3x-11}{x^2+x+2}+\frac{-3x+16}{x^2+1}=\frac{3x+11}{x^2+x+2}-\frac{3x-16}{x^2+1}
    \end{equation*}
\end{document}
