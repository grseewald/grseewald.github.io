\documentclass[11pt,a4paper]{article}

\usepackage[utf8]{inputenc}
\usepackage[ngerman]{babel}

\usepackage{amsmath}
\usepackage{amssymb}
\usepackage{mathtools}
\usepackage{xfrac}
\usepackage{marvosym}

\usepackage{enumerate}

\usepackage{fancyhdr}

\usepackage{array}   % for \newcolumntype macro
\newcolumntype{L}{>{$}l<{$}} % math-mode version of "l" column type
\newcommand{\Aufgabe}[1]{\section*{Aufgabe #1}}


\pagestyle{fancy}

\fancyhf[HR]{Lukas Vormwald, Noah Mehling, Gregor Seewald}
\fancyhf[HL]{Übung:Dienstag 12:00}

\title{Mathematik für Informatiker 2\\Übungsblatt 7}
\author{Lukas Vormwald \and Noah Mehling \and Gregor Seewald}
\date{Übung 5:Dienstag 12:00}
\DeclarePairedDelimiter\ceil{\lceil}{\rceil}
\DeclarePairedDelimiter\floor{\lfloor}{\rfloor}

\DeclarePairedDelimiter\abs{\lvert}{\rvert}%
\DeclarePairedDelimiter\norm{\lVert}{\rVert}%

% Swap the definition of \abs* and \norm*, so that \abs
% and \norm resizes the size of the brackets, and the
% starred version does not.
\makeatletter
\let\oldabs\abs
\def\abs{\@ifstar{\oldabs}{\oldabs*}}
%
\let\oldnorm\norm
\def\norm{\@ifstar{\oldnorm}{\oldnorm*}}
\makeatother

\begin{document}

  \maketitle

  %\section*{Aufgabe 1}
  \Aufgabe{1}

    \begin{enumerate}[a)]
      \item
        \begin{align*}
          z^5&=16\left(1-\sqrt{3}i\right)\\
          z&=\sqrt[5]{16\left(1-\sqrt{3}i\right)}\\
          z&=\sqrt[5]{16}\sqrt[5]{1-\sqrt{3}i}
        \end{align*}
      \item
        \begin{align*}
          iz^2+(1-i)z-3&=0\\
          \text{Substitution:}z&=x\\
          ix^2+(1-i)x-3&=0\\
          x_{1,2}&=\frac{-1+i\pm\sqrt{\left(1-i\right)^2-4\cdot i \cdot (-3)}}{2i}\\
          &=\frac{-1+i\pm\sqrt{1-2i-1+12i}}{2i}\\
          &=\frac{-1+i\pm\sqrt{10i}}{2i}\\
          z_{1,2}&=\frac{-1+i\pm\sqrt{10i}}{2i}
        \end{align*}
      \item
        \begin{align*}
          z^4+z^2+iz^2+i&=0\\
          \text{Substitution:}z^2&=x\\
          x^2+\underbrace{x+ix}_{(i+1)x}+i&=0
        \end{align*}
        \begin{align*}
          x_{1,2}&=\frac{-i-1\pm\sqrt{-1+2i+1-4\cdot 1 \cdot i}}{2}\\
          &=\frac{-i-1\pm\sqrt{-2i}}{2}\\
          &=\frac{-i-1\pm\sqrt{-1\cdot 2i}}{2}\\
          &=\frac{-i-1\pm i \cdot \sqrt{2i}}{2}\\
          &=\frac{-i-1\pm\left(i+2i\right)}{2}\\
          &=\frac{2i-1}{2},\frac{-4i-1}{2}
        \end{align*}
        \begin{align*}
          z_{1,2}&=\pm\sqrt{\frac{2i-1}{2}}
        \end{align*}
    \end{enumerate}

  \section*{Aufgabe 2}

    $\frac{1}{1-z-z^2-z^3}$\\
    Nullstellen von Nennerpolynom:\\
    $z=1$\\
    $(z^3-z^2-z+1) : (z-1)=z^2-1$\\
    $\Rightarrow z^2=1 \Rightarrow z_2=1,z_3=-1$\\
    $\Rightarrow \frac{1}{\left(z-1\right)^2\cdot\left(z+1\right)}=\frac{A}{\left(z+1\right)}+\frac{B}{\left(z-1\right)}+\frac{C}{\left(z-1\right)^2}$\\
    $\Rightarrow 1=\frac{A\cdot \left(z-1\right)^2-\left(z+1\right)}{\left(z+1\right)}+\frac{B\cdot \left(z-1\right)^2\cdot \left(z+1\right)}{\left(z-1\right)}+\frac{C\cdot \left(z-1\right)^2\cdot\left(z+1\right)}{\left(z-1\right)^2}$\\
    $=A\cdot \left(z-1\right)^2+B\cdot\left(z-1\right)\left(z+1\right)+C\cdot\left(z+1\right)$\\
    $\Rightarrow z=1 \Rightarrow 1=2C \Rightarrow C=0,5$\\
    $z=-1 \Rightarrow 1=4A \Rightarrow A=0,25$\\
    $\Rightarrow$ z beliebig, z.B $0$
    \begin{align*}
      \Rightarrow 1 =& 0,25 \cdot 1 + B \cdot \left( -1 \right) \cdot 1 + 0,5 \cdot 1\\
      1 =& 0,25 - B + 0,5\\
      B =& 0,75 - 1 + 0,5 &\Rightarrow B = -0,25
    \end{align*}
    $\Rightarrow \frac{1}{1-z-z^2+z^3} = \frac{0,25}{\left(z+1\right)} - \frac{0,25}{\left(z-1\right)}+ \frac{0,5}{\left(z-1\right)^2}$

    \newpage

  \section*{Aufgabe 3}

    \begin{enumerate}[a)]
      \item $\frac{1}{z^3-iz^2-z+i}$\\
      $z=1:1^3-i-1+i=0?$\\
      $\left(z^3-iz^2-z+i\right):\left(z-1\right)=z^2-iz+z-i$
      \begin{align*}
        z_{1/2}&=\frac{-\left(-i+1\right)\pm\sqrt{\left(-i+1\right)^2-4-\left(-i\right)}}{2}\\
        &=\frac{i-1\pm\sqrt{1-2i-1+4i}}{2}\\
        &=\frac{i-1\pm\sqrt{2i}}{2}\\
        &=\frac{i-1\pm\left(i+1\right)}{1}\\
        z_1&=\frac{2i}{2}=i\\
        z_2&=\frac{-2}{2}=-1
      \end{align*}
      $\frac{a}{z-1}+\frac{b}{z-i}+\frac{c}{z+1}=\frac{1}{z^3-iz^2-z+i}=\frac{1}{\left(z-1\right)\left(z-i\right)\left(z+1\right)}$\\
      $\left(\frac{a}{z-1}+\frac{b}{z-i}+\frac{c}{z+1}\right)\cdot\left(z^3-iz^2-z+i\right)=1$\\
      $a-\left(z-i\right)\left(z+1\right)+b\left(z-1\right)\left(z+1\right)+c\left(z-1\right)\left(z-i\right)=1$\\
      Sei $z=1$
      \begin{align*}
        a\cdot\left(1-i\right)\left(2\right)+b\left(0\right)\left(2\right)+c\left(0\right)\left(1-i\right)&=1\\
        2a\left(1-i\right)&=1\\
        2a-2ai&=1\\
        a&=\frac{1}{2\left(1-i\right)}=\frac{1}{2-2i}
      \end{align*}
      Sei $z=i$
      \begin{align*}
        a\cdot\left(0\right)\left(2\right)+b\left(i-1\right)\left(i+1\right)+c\left(2\right)\left(0\right)&=1\\
        b\left(i-1\right)\left(i+1\right)&=1\\
        b&=\frac{1}{\left(i-1\right)\left(i+1\right)}
      \end{align*}
      Sei $z=-1$
      \begin{align*}
        a\cdot\left(z\right)\left(0\right)+b\cdot\left(z\right)\left(0\right)+c\left(-z\right)\left(-1-i\right)&=1\\
        -2c\left(-1-i\right)&=1\\
        c&=\frac{1}{\left(-z\right)\left(-1-i\right)}=\frac{1}{z+zi}\\
        &=\frac{\frac{1}{z-zi}}{z-1}+\frac{\frac{1}{\left(i-1\right)\left(i+1\right)}}{z-i}+\frac{\frac{1}{z+zi}}{z+1}
        \\&=\frac{1}{(z-zi)(z-1)+(i-1)(i+1)(z-i)+(z+zi)(z+1)}\\
    \end{align*}
    \begin{equation*}
        =\frac{1}{\left(z-zi\right)\left(z-1\right)}+\frac{1}{\left(i-1\right)\left(i+1\right)\left(z-i\right)}+\frac{1}{\left(z+zi\right)\left(z+1\right)}
    \end{equation*}
      \item $\frac{z-1}{z^4+z^2}$\\
      $z=0$:$0+0=0$\\
      \begin{tabular}{lr}
        $z^2=x$&$z_1=0$\\
        $x^2+x$&$z_{2/3}\pm\sqrt{-1}$\\
        $x\left(x+1\right)=0$ & $z_2=i;z_3=-i$
      \end{tabular}
      \begin{equation*}
        \frac{a}{z-0}+\frac{b}{\left(z-0\right)^2}+\frac{c}{z-i}+\frac{d}{z+i} = \frac{z-1}{z^4+z^2} = \frac{-1}{\left(z-0\right)^2\left(z-i\right)\left(z+i\right)}
      \end{equation*}
      \begin{equation*}
        a\left(z\right)\left(z-i\right)\left(z+i\right)+b\left(z-i\right)\left(z+i\right)+c\left(z\right)^2\left(z+i\right)+d\left(z\right)^2\left(z-i\right)=z-1
      \end{equation*}
      Sei $z=0$
      \begin{align*}
        a\cdot 0+b\left(-i\right)\left(i\right)+c\cdot 0+ d\cdot 0 &= -1\\
        b&=\frac{-1}{\left(-i\right)\left(i\right)}&=-1
      \end{align*}
      Sei $z=i$
      \begin{align*}
        a\cdot 0+b\cdot 0+c\cdot i^2 \cdot \left(2i\right)+d\cdot 0&=i-1\\
        c&=\frac{i-1}{-\left(2i\right)}
      \end{align*}
      Sei $z=-i$
      \begin{align*}
        a\cdot 0+b\cdot 0+c\cdot 0+d\left(-i\right)^2\left(-2i\right)&=-i-1\\
        d&=\frac{-i-1}{2i}
      \end{align*}
      Sei $z=1$
      \begin{align*}
        a(1-i)(1+i)+(-1)(1-i)(1+i)+\frac{i-1}{-(2i)}(1+i)+\frac{-i-1}{2i}(1-i)&=0\\
        a(1-i)(1+i)+(-1+i)(1+i)+\frac{(i-1)(1+i)}{-(2i)}+\frac{(-i-1)(1-i)}{2i}&=0\\
        a\cdot2-2+2\cdot\frac{(i-1)(1+i)}{-2i}&=0\\
    \end{align*}
    \begin{align*}
        \Rightarrow 2a-2+2\frac{-2}{-2i}&=0\\
        \Rightarrow 2a-2+\frac{2}{i}&=0\\
        \Rightarrow 2a&=2-\frac{2}{i}\\
        \Rightarrow a&=\frac{2-\frac{2}{i}}{2}&=\frac{2-2\left(\frac{1}{i}\right)}{2}=1-\frac{1}{i}
    \end{align*}
    \begin{align*}
      \Rightarrow \frac{z-1}{\left(z-0\right)^2\left(z-i\right)(z+i)}&=\frac{1-\frac{1}{i}}{z-0}+\frac{-1}{(z-0)^2}+\frac{i-1}{-2i}\frac{1}{z-i}+\frac{-i-1}{2i}\frac{1}{z+i}\\
      \Rightarrow \frac{z-1}{(z-0)^2(z-i)(z+i)}&=\frac{1-\frac{1}{i}}{z}-\frac{1}{z^2}+\frac{i-1}{-2iz-2}+\frac{-i-1}{2iz-2}
    \end{align*}
    \end{enumerate}

  \section*{Aufgabe 4}

    $\frac{2x^2 + 7x + 5}{\left( x^2 + x + 2\right)\left( x^2 +1 \right)}$\\
    Nullstellen:\\
    $\left(x^2+1\right)\Rightarrow$ keine\\
    $x^2+x+2 \Rightarrow x_{1,2} = \frac{-1\pm\sqrt{1^2-4\cdot1\cdot2}}{2-1}$\qquad\Lightning keine Nullstelle\\
    $\Rightarrow \frac{Ax+B}{4x^2+x+2} + \frac{Cx+D}{x^2+1}$\\
    $\Rightarrow \frac{2x^2+7x+5}{\left(x^2+x+2\right)\left(x^2+1\right)} = \frac{Ax+B}{x^2+x+2} + \frac{Cx+D}{x^2+1}$
    \begin{align*}
      \frac{2x^2+7x+5}{\left(x^2+x+2\right)\left(x^2+1\right)} =& \frac{\left(Ax+B\right)\left(x^2+1\right)}{\left(x^2+x+2\right)\left(x^2+1\right)} + \frac{\left(Cx+D\right)\left(x^2+x+2\right)}{\left(x^2+x+2\right)\left(x^2+1\right)}\\
      =& \frac{\left(Ax+B\right)\left(x^2+1\right) + \left(Cx + D\right)\left(x^2+x+2\right)}{\left(x^2+x+2\right)\left(x^2+1\right)}\\
      =& \frac{Ax^3+Ax+Bx^2+B+Cx^3+Cx^2+2Cx+Dx^2+Dx+D}{\left(x^2+x+2\right)\left(x^2+1\right)}\\
    \frac{2x^2+7x+5}{\left(x^2+x+2\right)\left(x^2+1\right)} =& \frac{x^3\left(A+C\right)+x^2\left(B+C+D\right)+x\left(A+2C+D\right)+\left(B+D\right)}{\left(x^2+x+2\right)\left(x^2+1\right)}
    \end{align*}
    \begin{align}
      \Rightarrow A+C=&0 & \Rightarrow A=&C \label{eq:1}\\
      B+C+D=&2 \label{eq:2}\\
      A+2C+D=&7 \label{eq:3}\\
      B+D=&5 \label{eq:4}
    \end{align}
    \begin{align}
      \Rightarrow \eqref{eq:1} \text{ in } \eqref{eq:3} \Rightarrow A+2A+D=7 \Rightarrow 3A+D=7 \Rightarrow D=7-3A \label{eq:3*}\\
      \Rightarrow \eqref{eq:1} \& \eqref{eq:3*} \text{ in } \eqref{eq:2} \Rightarrow B+A+7-3A=2 \Rightarrow B-2A=-5 \Rightarrow B=2A-5 \label{eq:2*}\\
      \Rightarrow \eqref{eq:2*} \& \eqref{eq:3*} \text{ in } \eqref{eq:4} \Rightarrow 2A-5+7-3A=5 \Rightarrow 2A-3A=3 \Rightarrow -A=3 \Rightarrow A=-3
    \end{align}
    \begin{align*}
      B&=2\cdot \left(-3\right)-5 &= 11\\
      C&=A&=-3\\
      D&=7-3\cdot\left(-3\right) &=7+9 &=16
    \end{align*}
      $\Rightarrow \frac{2x^2+7x+5}{\left(x^2+x+2\right)\left(x^2+1\right)}=\frac{-3x-11}{x^2+x+2}+\frac{-3x+16}{x^2+1}=\frac{3x+11}{x^2+x+2}-\frac{3x-16}{x^2+1}$
\end{document}
