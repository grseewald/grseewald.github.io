\documentclass[11pt,a4paper]{article}

\usepackage[utf8]{inputenc}
\usepackage[ngerman]{babel}

\usepackage{amsmath}
\usepackage{amssymb}
\usepackage{mathtools}
\usepackage{xfrac}
\usepackage{marvosym}

\usepackage{enumerate}

\usepackage{fancyhdr}

\usepackage{array}   % for \newcolumntype macro
\newcolumntype{L}{>{$}l<{$}} % math-mode version of "l" column type
\newcommand{\Aufgabe}[1]{\section*{Aufgabe #1}}


\pagestyle{fancy}

\fancyhf[HR]{Lukas Vormwald, Noah Mehling, Gregor Seewald}
\fancyhf[HL]{Übung:Dienstag 12:00}

\title{Mathematik für Informatiker 2\\Übungsblatt 9}
\author{Lukas Vormwald \and Noah Mehling \and Gregor Seewald}
\date{Übung 5:Dienstag 12:00}
\DeclarePairedDelimiter\ceil{\lceil}{\rceil}
\DeclarePairedDelimiter\floor{\lfloor}{\rfloor}

\DeclarePairedDelimiter\abs{\lvert}{\rvert}%
\DeclarePairedDelimiter\norm{\lVert}{\rVert}%

% Swap the definition of \abs* and \norm*, so that \abs
% and \norm resizes the size of the brackets, and the
% starred version does not.
\makeatletter
\let\oldabs\abs
\def\abs{\@ifstar{\oldabs}{\oldabs*}}
%
\let\oldnorm\norm
\def\norm{\@ifstar{\oldnorm}{\oldnorm*}}
\makeatother

\begin{document}

\maketitle

  \Aufgabe{1}
    $P(x)=(x-x_0)\cdot q(x)=(x-x_0)(x-x_1)...(x-x_{j-1})(x-x_{j+1})...(x-x_n)\cdot a$\\
    $P(x_j)=1\Rightarrow(x_j-x_0)q(x)=\underbrace{(x_j-x_0)(x_j-x_1)...(x_j-x_{j-1})}_{\neq 0 \text{ da paarweise verschieden}}\cdot a=1$\\
    $\Rightarrow a=\frac{1}{(x-x_0)(g(x))}=$Lagrangebasis\\
    mit $a$ Polynom von Grad $n-n=0\Rightarrow$ Konstant
    \paragraph{2.Teil}
    Aufgrung von Lemma 14.1. lässt sich ein beliebiges Polynom $q(x)$ angeben.\\
    Nun stellt man eine Interpolationsaufgabe $q(x)$ mit $y_j=p(x_j)$\\
    Nach 14.1. lässt sich $P(x)$ schreiben als\\
    $P(x)=\sum\limits_{j=0}^{n}y_jI_\xi(x)\in\mathbb{P}_n$\\
    Nach Satz 14.1. ist diese Lösung eindeutig. Somit ist $P(x)=q(x)$ und $y_j$ eine Basis des Lösungsraums, da $I_j$ als linearkombination dargestellt werden kann.

    \newpage

  \Aufgabe{2}
    $P(x)=a_0+a_1x+a_2x^2+a_3x^3\qquad P'(x)=a_1+a_22x$\\
    $y_0=a_0$\\
    $y_1=y_0+a_1(x_1-x_0)$\\
    $y_2=a_1+a_22(x_1-x_0)$\\
    $y_3=y_0+a_1+a_2(x_1-x_0)(x_1)+a_3(x_2-x_1)(x_2-x_0)$\\
    $\Rightarrow a_0=y_0$\\
    $a_1=\frac{y_1-y_0}{(x_1-x_0)}$\\
    $a_2=\frac{y_2-\frac{y_1-y_0}{(x_1-x_0)}}{2x}$\\
    $a_3=\frac{y_3-y_0-\left(\frac{y_2-\frac{y_1-y_0}{(x_1-x_0)}}{2x}\right)(x_1-x_0)(x_1)}{(x_2-x_1)(x_2-x_0)}$

    Da $a_0 ... a_3$ Vorfaktoren der Standardbasis sind, ist dies ebenfalls eine Basis des $\mathbb{P}$

    \newpage

    \Aufgabe{3}

      \begin{enumerate}[a)]
        \item
        \begin{align*}
            q_0&=(x-1)^2\\
            q_1&=-(x-1)^2-1\\
            q_2&=(x-0,5)^2-0,25
        \end{align*}
        Zeigen, dass $q_0,q_1,q_2$ Basis von $\mathbb{P}_2$:\\
        $a\cdot(x-1)^2+b\cdot\left(-(x-1)^2+1\right)+c\cdot(x-0,5)^2-0,25=0\Rightarrow a,b,c=0$\\
        $a+b\cdot 0+c\cdot 0=0$\\
        $\Rightarrow a\stackrel{!}{=}0$\\
        \begin{align*}
          x&=1\\
          a\cdot 0+b-c&=0\\
          \Rightarrow b&=c\\
          \Rightarrow b&=0\\
          c&=0\\
        \end{align*}
        $\Rightarrow$ angegeben $q_0,q_1,q_2$ sind Basis.
        \item
        $P(x)=a_0+a_1x+a_2x^2\qquad P'(x)=a_1+a_22x$
        \begin{align*}
          3&=a_0\\
          2&=3+a_1+a_2 & 2&=3+4-2a_2+a_2\\
          4&=a_1+2a_2 & 3a_2=5\\
          a_1&=4-2a_2 & a_2=\frac{5}{3}\\
          &\Rightarrow a_1=4-\frac{10}{3}\approx 0,9
        \end{align*}
      \end{enumerate}
      \newpage

      \Aufgabe{4}

        \begin{enumerate}[a)]
          \item Sei $f: [a,b] \to {\bf R}$ $(n+1)$-mal stetig differenzierbar. Dann hat das Restglied $R_nf := f - L_n f$ bei der Polynominterpolation an $n+1$ paarweise verschiedenen Stützstellen $x_0,...,x_n \in [a,b]$ die Lagrangesche Darstellung
          \begin{equation*}
            (R_n f)(x) = \frac{f^{(n+1)} ( \xi )}{(n+1) !} \prod_{j=0}^n (x-x_j),\quad x \in [a,b]
          \end{equation*}
          mit einer von der Entwicklungsstelle $x$ abhängigen Stelle  $\xi \in [a,b]$.
          Ist x selbst Stützstelle, so ist die Aussage erfüllt. Sei nun
          \begin{equation*}
            \omega _{n+1} (x) := \prod_{j=0}^n (x-x_j)
          \end{equation*}
          und für festes, aber beliebiges $x \in [a,b], x \ne x_j,  j=0,...,n$ eine Hilfsfunktion $g: [a,b] \to {\bf R}$ definiert durch
          \begin{equation*}
            g(y):= f(y)-(L_n f)(y) - \omega _{n+1} (y) \frac{f(x)-(L_n f)(x)}{ \omega _{n+1}(x)},y \in [a,b] .
          \end{equation*}
          Nach den Voraussetzungen an $f$ ist $g$ $(n+1)$-mal stetig differenzierbar. Ferner hat $g$ die $n+2$ Nullstellen $x,x_0,\dots,x_n$. Der Satz von Rolle besagt nun, dass die Ableitung $g'$ zwischen zwei Nullstellen von $g$ wieder eine Nullstelle besitzt. Damit hat $g'$ $n+1$ paarweise verschiedene Nullstellen auf $[a,b]$. Sukzessive Wiederholung dieses Arguments ergibt, dass $g^{(n+1)}$ eine Nullstelle  $\xi$  in $[a,b]$ hat. Man berechnet
          \begin{equation*}
            0 = g^{(n+1)} ( \xi ) = f^{(n+1)} ( \xi ) - (n+1)! \frac{(R_n f)(x)}{ \omega_{n+1} (x) }
          \end{equation*}
          und erhält die Behauptung.
          \item Hier können wir die Formel aus a verwenden, da alle Vorraussetzungen erfüllt sind. Da wir 4 Stützstellen, also $n+1$ mit $n=3$, besitzen. Sie ist laut Aufgabenstellung $n+1$-mal stetig differenzierbar und die Stützstellen sind paarweise verschieden und gleichen einem $f(x_i)$ mit $i=0,\dots,3$
        \end{enumerate}

\end{document}
