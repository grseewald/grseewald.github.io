\documentclass[11pt,a4paper]{article}

\usepackage[utf8]{inputenc}
\usepackage[ngerman]{babel}

\usepackage{amsmath}
\usepackage{amssymb}
\usepackage{mathtools}
\usepackage{xfrac}
\usepackage{marvosym}

\usepackage{enumerate}

\usepackage{fancyhdr}

\usepackage{array}   % for \newcolumntype macro
\newcolumntype{L}{>{$}l<{$}} % math-mode version of "l" column type
\newcommand{\Aufgabe}[1]{\section*{Aufgabe #1}}


\pagestyle{fancy}

\fancyhf[HR]{Lukas Vormwald, Noah Mehling, Gregor Seewald}
\fancyhf[HL]{Übung:Dienstag 12:00}

\title{Mathematik für Informatiker 2\\Übungsblatt 9}
\author{Lukas Vormwald \and Noah Mehling \and Gregor Seewald}
\date{Übung 5:Dienstag 12:00}
\DeclarePairedDelimiter\ceil{\lceil}{\rceil}
\DeclarePairedDelimiter\floor{\lfloor}{\rfloor}

\DeclarePairedDelimiter\abs{\lvert}{\rvert}%
\DeclarePairedDelimiter\norm{\lVert}{\rVert}%

% Swap the definition of \abs* and \norm*, so that \abs
% and \norm resizes the size of the brackets, and the
% starred version does not.
\makeatletter
\let\oldabs\abs
\def\abs{\@ifstar{\oldabs}{\oldabs*}}
%
\let\oldnorm\norm
\def\norm{\@ifstar{\oldnorm}{\oldnorm*}}
\makeatother

\begin{document}

\maketitle

  \Aufgabe{1}
    $P(x)=(x-x_0)\cdot q(x)$

  \Aufgabe{2}
    $P(x)=a_0+a_1x+a_2x^2+a_3x^3\qquad P'(x)=a_1+a_22x$\\
    $y_0=a_0$\\
    $y_1=y_0+a_1(x_1-x_0)$\\
    $y_2=a_1+a_22(x_1-x_0)$\\
    $y_3=y_0+a_1+a_2(x_1-x_0)(x_1)+a_3(x_2-x_1)(x_2-x_0)$\\
    $\Rightarrow a_0=y_0$\\
    $a_1=\frac{y_1-y_0}{(x_1-x_0)}$\\
    $a_2=\frac{y_2-\frac{y_1-y_0}{(x_1-x_0)}}{2x}$\\
    $a_3=\frac{y_3-y_0-\left(\frac{y_2-\frac{y_1-y_0}{(x_1-x_0)}}{2x}\right)(x_1-x_0)(x_1)}{(x_2-x_1)(x_2-x_0)}$

    Da $a_0 ... a_3$ Vorfaktoren der Standardbasis sind, ist dies ebenfalls eine Basis des $\mathbb{P}$

    \Aufgabe{3}

      \begin{enumerate}[a)]
        \item
        \begin{align*}
            q_0&=(x-1)^2\\
            q_1&=-(x-1)^2-1\\
            q_2&=(x-0,5)^2-0,25\\
        \end{align*}
        Zeigen, dass $q_0,q_1,q_2$ Basis von $\mathbb{P}_2$:\\
        $a\cdot(x-1)^2+b\cdot\left(-(x-1)^2+1\right)+c\cdot(x-0,5)^2-0,25=0\Rightarrow a,b,c=0$\\
        $a+b\cdot 0+c\cdot 0=0$\\
        $\Rightarrow a\stackrel{!}{=}0$\\
        \begin{align*}
          x&=1\\
          a\cdot 0+b-c&=0\\
          \Rightarrow b&=c\\
          \Rightarrow b&=0\\
          c&=0\\
        \end{align*}
        $\Rightarrow$ angegeben $q_0,q_1,q_2$ sind Basis.
        \item
        $P(x)=a_0+a_1x+a_2x^2\qquad P'(x)=a_1+a_22x$\\
        \begin{align*}
          3&=a_0\\
          2&=3+a_1+a_2 & 2&=3+4-2a_2+a_2\\
          4&=a_1+2a_2 & 3a_2=5\\
          a_1&=4-2a_2 & a_2=\frac{5}{3}\\
          &\Rightarrow a_1=4-\frac{10}{3}\approx 0,9 
        \end{align*}
      \end{enumerate}

\end{document}
