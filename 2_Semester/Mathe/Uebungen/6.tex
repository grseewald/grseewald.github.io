\documentclass[11pt,a4paper]{article}

\usepackage[utf8]{inputenc}
\usepackage[ngerman]{babel}

\usepackage{amsmath}
\usepackage{amssymb}
\usepackage{mathtools}
\usepackage{xfrac}

\usepackage{enumerate}

\usepackage{fancyhdr}

\usepackage{array}   % for \newcolumntype macro
\newcolumntype{L}{>{$}l<{$}} % math-mode version of "l" column type


\pagestyle{fancy}

\fancyhf[HR]{Lukas Vormwald, Noah Mehling, Gregor Seewald}
\fancyhf[HL]{Übung:Dienstag 12:00}

\title{Mathematik für Informatiker 2\\Übungsblatt 5}
\author{Lukas Vormwald \and Noah Mehling \and Gregor Seewald}
\date{Übung 5:Dienstag 12:00}
\DeclarePairedDelimiter\ceil{\lceil}{\rceil}
\DeclarePairedDelimiter\floor{\lfloor}{\rfloor}

\DeclarePairedDelimiter\abs{\lvert}{\rvert}%
\DeclarePairedDelimiter\norm{\lVert}{\rVert}%

% Swap the definition of \abs* and \norm*, so that \abs
% and \norm resizes the size of the brackets, and the
% starred version does not.
\makeatletter
\let\oldabs\abs
\def\abs{\@ifstar{\oldabs}{\oldabs*}}
%
\let\oldnorm\norm
\def\norm{\@ifstar{\oldnorm}{\oldnorm*}}
\makeatother

\begin{document}
  \maketitle
  \section*{Aufgabe 1}
    \begin{enumerate}[a)]
      \item
        \begin{align*}
            f(x) =& 8x^7 + 2x + 10 \\
            8x^7 + 2x + 10 \stackrel{!}{=}& 10 \\
            8x^7 + 2x =& -10 \\
            x =& -1
        \end{align*}
        \begin{center}
          \begin{tabular}{cc}
            $\swarrow$&$\searrow$\\
            $f'(-2)=-1,018$&$f'(0)=10$\\
            negativ&positv\\
            $\searrow$&$\swarrow$\\
          \end{tabular}\\
          Kein Terrassenpunkt $\to$ 1 Extremstelle bei $x=-1$
        \end{center}
      \item
        \begin{align*}
          f'(x) =& e^{-x^2} \cdot (-2)x \\
          \underbrace{e^{-x^2}}_{\text{\tiny hat keine Nullstellen}} \cdot (-2) x \stackrel{!}{=} 0 \Leftrightarrow& -2x \stackrel{!}{=} 0\\
          x=&0
        \end{align*}
        \begin{center}
          \begin{tabular}{cc}
            $\swarrow$&$\searrow$\\
            $f'(-1)=0,74$&$f'(1)=0,74$\\
            positiv&negativ\\
            $\searrow$&$\swarrow$\\
          \end{tabular}\\
          Kein Terrassenpunkt $\to$ 1 Extremstelle bei $x=0$
        \end{center}
    \end{enumerate}

  \section*{Aufgabe 2}

    Sowohl $\sin(x)$ als auch $e^{(-x)}$ sind stetig, somit ist auch $\sin(x)-e^{(-x)}$ stetig.\\

    \begin{align*}
      f(0) &= \sin(0) - e^{0} = 0 - 1 = -1 \\
      f\left(\frac{\pi}{2}\right) &= \sin\left(\frac{\pi}{2}\right) - e^{\left(-\frac{\pi}{2}\right)} 1 > 1 - e^0 = 0 \Rightarrow f\left(\frac{\pi}{2}\right) > 0
    \end{align*}

    Somit hat die Funktion nach dem Mittelwertsatz mindestens eine Nullstelle in $\left[ 0 , \frac{\pi}{2} \right] \to $ Zeigen, dass genau eine Ableitung positiv ist:\\

    $f'(x) = \cos(x) - e^{(-x)} \cdot (- 1) = \cos(x) + e^{-x}$\\

    Da $e^{-x}$ immer positiv ist und der Cosinus im Intervall $\left[ 0 , \frac{\pi}{2} \right]$ zwischen $\left[ 1,0 \right]$ schwankt. ist die Ableitung im Intervall $\left[ 0 , \frac{\pi}{2} \right]$ nur positiv. Somit ist $f(x)$ im Intervall streng monoton steigend und kann nur eine Nullstelle im Intervall haben.

  \section*{Aufgabe 3}

  \begin{enumerate}[a)]
    \item Zu betrachten sind 3 Fälle:\\
          $f(x)=0$, Damit sind alle Funktionswerte $f(\phi) = f(x_1) = f(x_2)$. Somit sind diese Punkte relative Extrema der Funktion, jedoch kein striktes Extrema.\\
          Ist $f(x_1 + \phi)$ für ein kleines $\phi$ negativ, so gilt aufgrund ger Stetigkeit: $\lim\limits_{x \to \phi} f(x_1)=f(\phi)$. Da dieser Funktionswert nun negativ ist und diese Bedingung ebenfalls für $f(x_2 - \phi)$ gelten muss, so muss die Funktion ein lokales Extrema aufweisen, da diese sonst die Definition der Stetigkeit verletzen würde.\\
          Für $f(x_1+\phi)$ positiv verlaüft der Beweis analog zu $f(x_1+\phi)$ negativ.

    \item Es gibt Drei Fälle zu betrachten:
          \begin{enumerate}[1.]
            \item $f$ ist eine konstante Funktion. Dann ist $f'(x)=0$ und besitzt somit unendlich viele Nullstellen in $\mathbb{R}$. Somit besitzt auch $f$ höchstens unendlich viele Nullstellen ($\infty + 1$).

            \item $f$ ist ein Polynom. Dann gilt nach dem Fundamentalsatz der Algebra, dass der Grad des Polynoms die maximale Anzahl an Nullstellen angibt. Da die Ableitung von Polynomen der Form $f(x)=x^n\Rightarrow f'(x)=n\cdot x^{n-1}$ ist und somit auch der Grad der Ableitung der Funktion um 1 vermindert wird gilt die Aussage ebenfalls.

            \item $f$ ist eine trigonometrische Funktion: Da trigonometrische Funktionen ebenfalls unendlich viele Nullstellen aufweisen, gilt hier das selbe Argument wie bei 1.
          \end{enumerate}
  \end{enumerate}

  \section*{Aufgabe 4}


      \begin{tabular}{*{6}L}
        f'(x)&=\frac{1}{1+x}&+2\cos(x)&-3&-\frac{x}{2}&+x\cdot\frac{1}{2}\\
        f'(0)&=1&+2&-3&-0&+0\\
        =3&-3&=0&&&\to Extrapunkt?\\
        f''(x)&=\frac{-1}{\left( 1+x\right)^2}&+2\cdot\left( \sin(x) \right)&-\frac{1}{2}&+\frac{1}{2}\\
        f''(0)&=-1&+0&-\frac{1}{2}&+\frac{1}{2}&=-1\\
      \end{tabular}\\
      $\to$ kein Terassenpunkt bei $x=0$, da zweite Ableitung $\neq 0 \to$ Extrempunkt bei $x=0$

  \section*{Aufgabe 5}

    Die Funktion muss mindestens ein globales Minimum aufweisen. Da die Funktion stetig ist gilt:\\
    $\lim\limits_{x \to \phi} f(x)=f(\phi)$. Wählt man als $\phi$ nun $\infty$, so muss auch der Funktionswert gegen $\infty$ konvergieren. Aufgrund der Bedingung der Angabe jedoch gegen $-\infty$ ebenfalls.\\
    Nun existieren zwei Möglichkeiten:
    \begin{enumerate}
      \item Die gegebene Bedingung ist erfüllt, dann ist das gewählte $f(x)$ bereits das gesuchte Minimum.
      \item eine der beiden Bedingungen ist verletzt. Dann ist es möglich, eine Umgebung $f(x+\phi)$ bei Verletzung des $\lim\limits_{x \to +\infty}$, bzw. $f(x-\phi)$ bei Verletzung von $\lim\limits_{x \to -\infty}$ zu wählenm bis die Teilfolge nicht mehr gegen $f(x)$ konvergiert (man betrachtet also $\lim\limits_{x+\phi \to x} f(x)$). Dieser Wert entspricht dann Fall 1.
    \end{enumerate}

\end{document}
