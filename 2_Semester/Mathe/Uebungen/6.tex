\documentclass[11pt,a4paper]{article}

\usepackage[utf8]{inputenc}
\usepackage[ngerman]{babel}

\usepackage{amsmath}
\usepackage{amssymb}
\usepackage{mathtools}
\usepackage{xfrac}

\usepackage{enumerate}

\usepackage{fancyhdr}

\usepackage{array}   % for \newcolumntype macro
\newcolumntype{L}{>{$}l<{$}} % math-mode version of "l" column type


\pagestyle{fancy}

\fancyhf[HR]{Lukas Vormwald, Noah Mehling, Gregor Seewald}
\fancyhf[HL]{Übung:Dienstag 12:00}

\title{Mathematik für Informatiker 2\\Übungsblatt 5}
\author{Lukas Vormwald \and Noah Mehling \and Gregor Seewald}
\date{Übung 5:Dienstag 12:00}
\DeclarePairedDelimiter\ceil{\lceil}{\rceil}
\DeclarePairedDelimiter\floor{\lfloor}{\rfloor}

\DeclarePairedDelimiter\abs{\lvert}{\rvert}%
\DeclarePairedDelimiter\norm{\lVert}{\rVert}%

% Swap the definition of \abs* and \norm*, so that \abs
% and \norm resizes the size of the brackets, and the
% starred version does not.
\makeatletter
\let\oldabs\abs
\def\abs{\@ifstar{\oldabs}{\oldabs*}}
%
\let\oldnorm\norm
\def\norm{\@ifstar{\oldnorm}{\oldnorm*}}
\makeatother

\begin{document}
  \maketitle
  \section*{Aufgabe 1}
    \begin{enumerate}[a)]
      \item
        \begin{align*}
            f(x) =& 8x^7 + 2x + 10 \\
            8x^7 + 2x + 10 \stackrel{!}{=}& 10 \\
            8x^7 + 2x =& -10 \\
            x =& -1
        \end{align*}
        \begin{center}
          \begin{tabular}{cc}
            $\swarrow$&$\searrow$\\
            $f'(-2)=-1,018$&$f'(0)=10$\\
            negativ&positv\\
            $\searrow$&$\swarrow$\\
          \end{tabular}\\
          Kein Terrassenpunkt $\to$ 1 Extremstelle bei $x=-1$
        \end{center}
      \item
        \begin{align*}
          f'(x) =& e^{-x^2} \cdot (-2)x \\
          \underbrace{e^{-x^2}}_{\text{\tiny hat keine Nullstellen}} \cdot (-2) x \stackrel{!}{=} 0 \Leftrightarrow& -2x \stackrel{!}{=} 0\\
          x=&0
        \end{align*}
        \begin{center}
          \begin{tabular}{cc}
            $\swarrow$&$\searrow$\\
            $f'(-1)=0,74$&$f'(1)=0,74$\\
            positiv&negativ\\
            $\searrow$&$\swarrow$\\
          \end{tabular}\\
          Kein Terrassenpunkt $\to$ 1 Extremstelle bei $x=0$
        \end{center}
    \end{enumerate}

  \section*{Aufgabe 2}

  \section*{Aufgabe 3}

  \section*{Aufgabe 4}


      \begin{tabular}{*{6}L}
        f'(x)&=\frac{1}{1+x}&+2\cos(x)&-3&-\frac{x}{2}&+x\cdot\frac{1}{2}\\
        f'(0)&=1&+2&-3&-0&+0\\
        =3&-3&=0&&&\to Extrapunkt?\\
        f''(x)&=\frac{-1}{\left( 1+x\right)^2}&+2\cdot\left( \sin(x) \right)&-\frac{1}{2}&+\frac{1}{2}\\
        f''(0)&=-1&+0&-\frac{1}{2}&+\frac{1}{2}&=-1\\
      \end{tabular}
      $\to$ kein Terassenpunkt bei $x=0$, da zweite Ableitung $\neq 0 \to$ Extrempunkt bei $x=0$

\end{document}
