\documentclass[11pt,a4paper]{article}

\usepackage[utf8]{inputenc}
\usepackage[ngerman]{babel}

\usepackage{amsmath}
\usepackage{amssymb}
\usepackage{mathtools}
\usepackage{xfrac}
\usepackage{marvosym}

\usepackage{enumerate}

\usepackage{fancyhdr}

\usepackage{tikz}

\usepackage{stmaryrd}

\usepackage{array}   % for \newcolumntype macro
\newcolumntype{L}{>{$}l<{$}} % math-mode version of "l" column type
\newcommand{\Aufgabe}[1]{\section*{Aufgabe #1}}
\newcommand{\lb}[0]{\ensuremath{\left\lbrace}}
\newcommand{\rb}[0]{\ensuremath{\right\rbrace}}


\pagestyle{fancy}

\fancyhf[HR]{Lukas Vormwald, Noah Mehling, Gregor Seewald}
\fancyhf[HL]{Übung:Dienstag 12:00}

\title{Mathematik für Informatiker 2\\Übungsblatt 10}
\author{Lukas Vormwald \and Noah Mehling \and Gregor Seewald}
\date{Übung 5:Dienstag 12:00}
\DeclarePairedDelimiter\ceil{\lceil}{\rceil}
\DeclarePairedDelimiter\floor{\lfloor}{\rfloor}

\DeclarePairedDelimiter\abs{\lvert}{\rvert}%
\DeclarePairedDelimiter\norm{\lVert}{\rVert}%

% Swap the definition of \abs* and \norm*, so that \abs
% and \norm resizes the size of the brackets, and the
% starred version does not.
\makeatletter
\let\oldabs\abs
\def\abs{\@ifstar{\oldabs}{\oldabs*}}
%
\let\oldnorm\norm
\def\norm{\@ifstar{\oldnorm}{\oldnorm*}}
\makeatother

\begin{document}

\maketitle

  \Aufgabe{1}
    $\Omega=\lb(i_1,i_2,i_3)|1 \leq i_j \leq 10 \text{ für }j=1,2,3\rb$
  \begin{enumerate}[a)]
    \item $\abs{\Omega}=10^3=100$\\
    $P=\frac{1}{100}$
  \item $E_1:=\lb (i_1,i_2,i_3)|1\leq i_j\leq 3 \text{ für } j=1,2,3\rb$\\
  $\null \qquad P(E_1)=\frac{3}{10} \cdot \frac{3}{10} \cdot \frac{3}{10}=0,027=2,7\%$\\

  $E_2:=\lb (i_1,i_2,i_3)| 1 \leq i_1 \leq 3, 4 \leq j_i \leq 10 \text{ für } j=2,3\rb$\\
  $\null \qquad P(E_2)=\frac{3}{10} \cdot \frac{7}{10} \cdot \frac{7}{10}=0,147=14,7\%$\\

  $E_3:=\lb (i_1,i_2,i_3) | 1 \leq i_1 \leq 3 \rb$\\
  $\null \qquad P(E_3)=\frac{3}{10} \cdot 1 \cdot 1=0,3=30\%$\\

  $E_4:=\lb (i_1,i_2,i_3) | 1 \leq i_2 \leq 3 \rb$\\
  $\null \qquad P(E_4)=P(E_3)=30\%$\\

  \item $P(E_1 \cup E_2) = P(E_3) = 30\%$\\
  $P(E_2 \cup E_3) = P(E_3) = 30 \%$\\
  $F:=\lb (i_1,i_2,i_3) | 1 \leq i_1 \leq 3 \text{ oder } 1 \leq i_2 \leq 3 \text{ oder } 1 \leq i_3 \leq 3\rb$\\
  $P(F)=\frac{3}{10} \cdot 1 \cdot 1 + 1 \cdot \frac{3}{10} \cdot 1 + 1 \cdot 1 \cdot \frac{3}{10} = 0,3 + 0,3 + 0,3 = 0,9 = 90\%$
  \newpage
  \item $P(E_i \cap E_j) = P(E_i) \cdot P(E_j) \qquad 1 \leq i,j \leq 4$\\
    \begin{align*}
      E_1\cap E_2 &\lightning\\
      E_1\cap E_3 &= E_1\\
      E_1\cap E_4 &= E_1\\
      E_2\cap E_3 &= E_2\\
      E_3\cap E_4 &&\rightarrow P(E_3\cap E_4)=\frac{3}{10}\cdot\frac{3}{10}\cdot 1 = \frac{9}{100} =30\% \cdot 30\%&= P(E_3)\cdot P(E_4)
    \end{align*}
  \end{enumerate}
  \Aufgabe{2}
    $\frac{ {10\choose 3} \cdot {20\choose 3} }{{30\choose 6}} + \frac{ {10\choose 4} \cdot {20\choose 2} }{{30\choose 6}} + \frac{ {10\choose 5} \cdot {20\choose 1} }{{30\choose 6}} + \frac{ {10\choose 6} \cdot {20\choose 0} }{{30\choose 6}} = \frac{2426}{7917} \approx 0,306 = 30,6\%$

  \Aufgabe{3}
    $\abs{B_K}=0,004N\qquad\abs{B_g}=0,996N$\\
    $A$ ist erkannte Bombe\\
    $(A \cap B_K) = 0,98\abs{B_K}$\\
    $P(B_K|A)=\frac{P(A \cap B_K)}{P(A)}=\frac{0,98 \cdot 0,004}{0,98 \cdot 0,004 + 0,01 \cdot 0,996}=\frac{98}{347}\approx0,243$

  \Aufgabe{4}
    \begin{align*}
      P(B)&=0,99\cdot 0,3 + 0,01 \cdot 0,5 = 0,302 =30,2 \%\\
      P(A \cap B)&= 0,99 \cdot 0,3 \cdot 0,8 + 0,01 \cdot 0,5 \cdot 0,2 = 0,2386 = 23,86 \%\\
      P(A|B)&=\frac{P(A \cap B)}{P(B)} = \frac{23,86 \%}{30,2 \%} \approx 79 \%
    \end{align*}

\end{document}
