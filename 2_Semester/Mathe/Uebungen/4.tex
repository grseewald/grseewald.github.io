\documentclass[11pt,a4paper]{article}

\usepackage[utf8]{inputenc}
\usepackage[ngerman]{babel}

\usepackage{amsmath}
\usepackage{amssymb}
\usepackage{mathtools}
\usepackage{xfrac}

\usepackage{enumerate}

\usepackage{fancyhdr}

\pagestyle{fancy}

\fancyhf[HR]{Lukas Vormwald, Noah Mehling, Gregor Seewald}
\fancyhf[HL]{Übung:Dienstag 12:00}

\title{Mathematik für Informatiker 2\\Übungsblatt 5}
\author{Lukas Vormwald \and Noah Mehling \and Gregor Seewald}
\date{Übung 5:Dienstag 12:00}
\DeclarePairedDelimiter\ceil{\lceil}{\rceil}
\DeclarePairedDelimiter\floor{\lfloor}{\rfloor}

\DeclarePairedDelimiter\abs{\lvert}{\rvert}%
\DeclarePairedDelimiter\norm{\lVert}{\rVert}%

% Swap the definition of \abs* and \norm*, so that \abs
% and \norm resizes the size of the brackets, and the 
% starred version does not.
\makeatletter
\let\oldabs\abs
\def\abs{\@ifstar{\oldabs}{\oldabs*}}
%
\let\oldnorm\norm
\def\norm{\@ifstar{\oldnorm}{\oldnorm*}}
\makeatother

\begin{document}
	\maketitle
	
	\section*{Aufgabe 1}
		$x \leq 0 \Rightarrow f \left( x\right ) = 0 \Rightarrow f'\left( x \right)=???$\\
		$x > 0 \Rightarrow f \left( x \right) = x \cdot \sqrt{x} \Rightarrow f'\left( x \right) = \sqrt[\leftroot{-2}\uproot{5}\frac{3}{2}]{x}$
	\section*{Aufgabe 2}
		$\frac{ f \left( x \right) - f \left( 0 \right) }{x-0}=\abs{ \frac{f \left( x \right)}{x}}<x$\\
		$\Rightarrow$ Beide Seiten gehen gegen 0 $\rightarrow$ Differenzierbar
	\section*{Aufgabe 3}
		Angenommen $x \geq y \Rightarrow$ Kein Betrag an $\ln$ , da positiv\\
		$\Rightarrow$\\
			\begin{align*}
				\ln \frac{1 + e^x }{1+ e^y} &\leq x-y \\
				\frac{1 + e^x }{1+ e^y} &\leq e^{x-y} \\
				1 + e^x &\leq \left( e^{x-y} \right) \left( 1 + e^y \right)\\
				&\leq e^{x-y} + e^x\\
				&\geq 1
			\end{align*}
		ist $y > x$, so ist 
			\begin{equation*}
				\abs{ \ln \frac{1 + e^x }{1+ e^y} } = \abs{ \left( \ln \left( 1 + e^x\right) \right) - \left( \ln \left( 1 + e^y \right) \right)} = \abs{ \left( \ln \left( 1 + e^y \right) \right) - \left( \ln \left( 1 + e^x\right) \right)}
			\end{equation*}
	\section*{Aufgabe 4}
		$f$ ist zwei mal differenzierbar.\\
		$\rightarrow$ Satz von Taylor: $f$ ist $n+1$ mal differenzierbar mit $n+1$. Sei der Entwicklungspunkt $a=0$.\\
		$f(x)=f(a)+f'(a)(x-a)$\\ 
		$f(x)=f(0)+f'(0)(x)$ \qquad Mit der Vorraussetzung $f(0)=f'(0)=0$\\
		$f(x)=0$.\\
		Damit kann $f(x)$ durch das "Polynom" 0 dargestellt werden mit dem Restglied nach Lagrange:\\
		$R_1(x;0)=\frac{\left( x - 0 \right)^2}{ 2 ! } \cdot f''\left( \xi \right) = \frac{f''\left( \xi \right)}{2} \cdot x^2$\\
		wähle $c \geq \frac{f''\left( \xi\right)}{2}$, so ist $\abs{ f\left( x\right)}\leq c \cdot x^2$  
	\section*{Aufgabe 5}
		$\ln \left( 1 + h \right) = \ln 1 + \frac{1}{1}h + \frac{1}{2}\frac{h^2}{\left( 1 + \varepsilon \right)^2}$\\
		\begin{tabular}{lr}
			$\Rightarrow h = 0,1 \Rightarrow x = 1 - h - \frac{1}{2} \frac{h^2}{\left( 1+ \varepsilon \right)^2}$&für $0 \leq \varepsilon \leq 1$\\
		\end{tabular}
		$\Rightarrow x = 1 - 0,1 - \frac{1}{2} \frac{0,1}{\left( 1 + 0,1\right)^2} = 0,95$
	\section*{Aufgabe 6}
		\begin{enumerate}[a)]
			\item
			
			\item
		\end{enumerate}
\end{document}