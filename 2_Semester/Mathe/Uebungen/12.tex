\documentclass[11pt,a4paper]{article}
\usepackage[utf8]{inputenc}
\usepackage[ngerman]{babel}
%\usepackage[official]{eurosym}

\usepackage{amsmath}
\usepackage{amssymb}
\usepackage{mathtools}
\usepackage{xfrac}
\usepackage{marvosym}

\usepackage{enumerate}

\usepackage{fancyhdr}

\usepackage{tikz}

\usepackage{array}   % for \newcolumntype macro
\newcolumntype{L}{>{$}l<{$}} % math-mode version of "l" column type
\newcommand{\Aufgabe}[1]{\section*{Aufgabe #1}}
\newcommand{\lb}[0]{\ensuremath{\left\lbrace}}
\newcommand{\rb}[0]{\ensuremath{\right\rbrace}}


\pagestyle{fancy}

\fancyhf[HR]{Lukas Vormwald, Noah Mehling, Gregor Seewald}
\fancyhf[HL]{Übung:Dienstag 12:00}

\author{Lukas Vormwald \and Noah Mehling \and Gregor Seewald}
\date{Übung 5:Dienstag 12:00}
\DeclarePairedDelimiter\ceil{\lceil}{\rceil}
\DeclarePairedDelimiter\floor{\lfloor}{\rfloor}

\newcommand{\Uebung}[1]{\title{Mathematik für Informatiker\\ Übungsblatt #1}}

\DeclarePairedDelimiter\abs{\lvert}{\rvert}%
\DeclarePairedDelimiter\norm{\lVert}{\rVert}%

% Swap the definition of \abs* and \norm*, so that \abs
% and \norm resizes the size of the brackets, and the
% starred version does not.
\makeatletter
\let\oldabs\abs
\def\abs{\@ifstar{\oldabs}{\oldabs*}}
%
\let\oldnorm\norm
\def\norm{\@ifstar{\oldnorm}{\oldnorm*}}
\makeatother

\Uebung{12}

\begin{document}
  \maketitle

    \Aufgabe{1}

      \begin{align*}
        \frac{e^{-a_y}}{k!}\cdot a_y^k \cdot \frac{e^{-a_x}}{k!} \cdot a_x^k &= \frac{e^{-(a_x+a_y)}}{k!} \cdot (a_x a_y)^k\\
        \frac{e^{-a_y}\cdot a_y^k \cdot e^{-a_x}\cdot a_x^k}{k!} &= \frac{e^{-(a_x+a_y)}}{k!} \cdot (a_x a_y)^k\\
        \frac{e^{-(a_y+a_x)}\cdot a_y^k \cdot a_x^k}{k!} &= \frac{e^{-(a_x+a_y)}}{k!} \cdot (a_x-a_y)^k\\
      \end{align*}

    \Aufgabe 2

      Parameter an Tag 1: $\left(n=1\right)=\alpha$\\
      $n$ Tage $\rightarrow$ Parameter $=n\cdot\alpha$\\
      Somit ist die Wahrscheinlichkeit\\
      $\frac{e^{-\left(\alpha\cdot n\right)}}{k!}\cdot \left(\alpha\cdot n\right)^k$

    \Aufgabe 3

      \begin{enumerate}
        \item Eine gute Abschätzung für $x_n$ ist das arithmetsche Mittel, also $\frac{x_1+x_2+\dots+x_n}{n}$, da es gegen p konvergiert.
        \item Es handelt sich um eine Konfidenzintervallaufgabe.\\
        Sei $\hat{p}$ die Abschätzung des Chefs.\\
        $\gamma=95\%$\\
        Sei $\alpha=1-\gamma=5\%$\\
        Nach der einfachen Approximation durch die Normalverteilung ist das Konfidenzintervall\\
        \begin{equation*}
          \left[ \hat{p} -\left(z_1-\frac{\alpha}{2}\right)\cdot \sqrt{\frac{\hat p \cdot \left(1-\hat{p}\right)}{n}}\quad ,\quad \hat p + \left(z_1-\frac{\alpha}{2}\right)\cdot \sqrt{\frac{\hat p \cdot \left(1-\hat{p}\right)}{n}} \right]
        \end{equation*}
        Die Intervallbreite darf nicht $>0,2$ sein, also gilt\\
        \begin{align*}
          2 \left(z_1-\frac{\alpha}{2}\right)\cdot \sqrt{\frac{\hat p \cdot \left(1-\hat{p}\right)}{n}} &\leq 0,2\\
          \frac{2\left(z_1-\frac{\alpha}{2}\right)}{0,2} &\leq \frac{1}{\sqrt{\frac{\hat p \cdot \left( 1- \hat p \right)}{n}}}\\
          \left( \frac{2\left(z_1-\frac{\alpha}{2}\right)}{0,2} \right)^2 & \leq \frac{1}{\frac{\hat p \cdot \left( 1- \hat p \right)}{n}}\\
          n &\geq \left( \frac{2\left(z_1-\frac{\alpha}{2}\right)}{0,2} \right)^2 \cdot \hat p \cdot \left( 1 - \hat p \right)
        \end{align*}
      \end{enumerate}

\end{document}
