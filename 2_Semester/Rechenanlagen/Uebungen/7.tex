\documentclass[11pt,a4paper]{article}

\usepackage[utf8]{inputenc}
\usepackage[ngerman]{babel}

\usepackage{amsmath}
\usepackage{amssymb}
\usepackage{mathtools}
\usepackage{xfrac}

\usepackage{color}
\usepackage[table]{xcolor}
\definecolor{blue}{RGB}{0,0,129}

\usepackage{enumerate}

\usepackage{fancyhdr}

\usepackage{graphicx}

\usepackage{listings}

\usepackage{tikz}

\pagestyle{fancy}

\fancyhf[HR]{Lukas Vormwald, Noah Mehling, Gregor Seewald}
\fancyhf[HL]{Übung:Dienstag 14:00}

\title{Rechenanlagen\\Übungsblatt 7}
\author{Lukas Vormwald \and Noah Mehling \and Gregor Seewald}
\date{Übung :Dienstag 14:00}

\newcommand{\nor}[0]{\ensuremath{\text{ nor }}}
\newcommand{\Aufgabe}[1]{\section{Aufgabe #1}}

\DeclarePairedDelimiter\ceil{\lceil}{\rceil}
\DeclarePairedDelimiter\floor{\lfloor}{\rfloor}

\DeclarePairedDelimiter\abs{\lvert}{\rvert}%
\DeclarePairedDelimiter\norm{\lVert}{\rVert}%

% Swap the definition of \abs* and \norm*, so that \abs
% and \norm resizes the size of the brackets, and the
% starred version does not.
\makeatletter
\let\oldabs\abs
\def\abs{\@ifstar{\oldabs}{\oldabs*}}
%
\let\oldnorm\norm
\def\norm{\@ifstar{\oldnorm}{\oldnorm*}}
\makeatother

\begin{document}

\maketitle

  \Aufgabe{1}

    \begin{align*}
      d&=\left(\overline{abc}\right)=\bar{a}\vee\bar{b}\vee\bar{c}\\
      e&=\left(\overline{ah}\right)=\bar{a}\vee\bar{h}\\
      f&=\left(\overline{bh}\right)=\bar{b}\vee\bar{h}\\
      g&=\left(\overline{ch}\right)=\bar{c}\vee\bar{h}\\
      h&=\left(\overline{defg}\right)\\
      &=\bar{d}\vee\bar{e}\vee\bar{f}\vee\bar{g}\\
      &=\left(abc\right)\vee\left(ah\right)\vee\left(bh\right)\vee\left(ch\right)\\
      &=\left(abc\right)\vee\left(a\vee b\vee c\right)h\\
    \end{align*}
    $\rightarrow$ stabil sind nur Belegungen mit $a,b,c=1$ oder $a,b,c=0$, da sonst der Wahrheitswert von h abhängt und somit nicht der Auswertung der Funktion $\left(abc\right)$ entspricht (immer $0$) für $a,b,c\neq 1,0$

  \Aufgabe{2}

    \input{tesseract/tesseract1.tex}
    \input{tesseract/tesseract2.tex}

  \Aufgabe{3}

    \begin{enumerate}[1.]
      \item
        \begin{tabular}{|cc|c|c|c|c|} \hline
          &a&0&0&1&1\\ \hline
          &b&0&1&1&0\\ \hline
          c&d&&&&\\ \hline
          0&0&&1&1&\\ \hline
          0&1&&&&\\ \hline
          1&1&1&1&&\\ \hline
          1&0&1&1&1&1\\ \hline
        \end{tabular}\\\\
        $c\bar{d}\vee\bar{a}cd\vee b\overline{cd}$
        \item
          \begin{tabular}{|cc|c|c|c|c|} \hline
            &a&0&0&1&1\\ \hline
            &b&0&1&1&0\\ \hline
            c&d&&&&\\ \hline
            0&0&1&&&1\\ \hline
            0&1&&1&&\\ \hline
            1&1&&&1&\\ \hline
            1&0&1&&1&1\\ \hline
          \end{tabular}\\\\
          $\overline{abd}\vee a\overline{bd}\vee abc \vee \bar{a}b\bar{c}d$
        \item
          \begin{tabular}{|cc|c|c|c|c|} \hline
            &a&0&0&1&1\\ \hline
            &b&0&1&1&0\\ \hline
            c&d&&&&\\ \hline
            0&0&1&1&1&\\ \hline
            0&1&1&1&1&1\\ \hline
            1&1&1&1&1&1\\ \hline
            1&0&&1&1&\\ \hline
          \end{tabular}\\\\
          $a\bar{b}d\vee\overline{ab}d\vee b \vee \overline{abcd}$
    \end{enumerate}
\end{document}
