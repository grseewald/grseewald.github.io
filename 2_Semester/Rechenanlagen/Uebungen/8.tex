\documentclass[11pt,a4paper]{article}

\usepackage[utf8]{inputenc}
\usepackage[ngerman]{babel}

\usepackage{amsmath}
\usepackage{amssymb}
\usepackage{mathtools}
\usepackage{xfrac}

\usepackage{color}
\usepackage[table]{xcolor}
\definecolor{blue}{RGB}{0,0,129}

\usepackage{enumerate}

\usepackage{fancyhdr}

\usepackage{graphicx}

\usepackage{listings}

\usepackage{tikz}

\usepackage{array}

\pagestyle{fancy}

\fancyhf[HR]{Lukas Vormwald, Noah Mehling, Gregor Seewald}
\fancyhf[HL]{Übung:Dienstag 14:00}

\title{Rechenanlagen\\Übungsblatt 8}
\author{Lukas Vormwald \and Noah Mehling \and Gregor Seewald}
\date{Übung :Dienstag 14:00}

\newcommand{\nor}[0]{\ensuremath{\text{ nor }}}
\newcommand{\Aufgabe}[1]{\section{Aufgabe #1}}

\DeclarePairedDelimiter\ceil{\lceil}{\rceil}
\DeclarePairedDelimiter\floor{\lfloor}{\rfloor}

\DeclarePairedDelimiter\abs{\lvert}{\rvert}%
\DeclarePairedDelimiter\norm{\lVert}{\rVert}%

% Swap the definition of \abs* and \norm*, so that \abs
% and \norm resizes the size of the brackets, and the
% starred version does not.
\makeatletter
\let\oldabs\abs
\def\abs{\@ifstar{\oldabs}{\oldabs*}}
%
\let\oldnorm\norm
\def\norm{\@ifstar{\oldnopackage namerm}{\oldnorm*}}
\makeatother

\begin{document}

\maketitle

  \Aufgabe{1}

   \begin{tabular}{cm{5cm}}

     \begin{tabular}{|c|c|c||c|}\hline
       B&L&R&K\\ \hline \hline
       0&0&0&0\\ \hline
       0&0&1&1\\ \hline
       0&1&0&1\\ \hline
       0&1&1&0\\ \hline
       1&0&0&0\\ \hline
       1&0&1&$x$\\ \hline
       1&1&0&$x$\\ \hline
       1&1&1&1\\ \hline

     \end{tabular}

     &

    %  \begin{figure}[bt]
    %    \centering
    %    \scalebox{0.6}
    %    {
        \begin{tikzpicture}[scale=5]
          \tikzstyle{vertex}=[circle,minimum size=20pt,inner sep=0pt]
          \tikzstyle{selected vertex} = [vertex, fill=red!24]
          \tikzstyle{other vertex} = [vertex, fill=blue!24]
          \tikzstyle{selected edge} = [draw,line width=5pt,-,red!50]
          \tikzstyle{edge} = [draw,thick,-,black]
          \node[other vertex] (v0) at (0,0) {$000$};
          \node[vertex] (v1) at (0,1) {$100$};
          \node[selected vertex] (v2) at (1,0) {$001$};
          \node[ other vertex] (v3) at (1,1) {$101$};
          \node[selected vertex] (v4) at (0.23, 0.4) {$010$};
          \node[other vertex] (v5) at (0.23,1.4) {$110$};
          \node[vertex] (v6) at (1.23,0.4) {$011$};
          \node[selected vertex] (v7) at (1.23,1.4) {$111$};
          \node[selected vertex] (v8) at (0,-0.5) {$1$};
          \node[other vertex] (v9) at (1,-0.5) {$*$};
          \draw[edge] (v0) -- (v1) -- (v3) -- (v2) -- (v0);
           \draw[edge] (v0) -- (v4) -- (v5) -- (v1) -- (v0);
           \draw[edge] (v2) -- (v6) -- (v7) -- (v3) -- (v2);
           \draw[edge] (v4) -- (v6) -- (v7) -- (v5) -- (v4);
        \end{tikzpicture}
%        }
%      \end{figure}

   \end{tabular}

   Primimplikanten:$\bar{L}R, L\bar{R}, BL, BR, \overline{BL}, \overline{BR}$\\
   \begin{align*}
     \Rightarrow K&= \bar{L}R\vee L\bar{R}\vee BL\vee BR\vee \overline{BL}\vee \overline{BR}\\
     &=\bar{L}R\vee L\bar{R}\vee B\left( L \vee R\right) \vee \bar{B}\left(\bar{L} \vee \bar{R}\right)
   \end{align*}


\end{document}
