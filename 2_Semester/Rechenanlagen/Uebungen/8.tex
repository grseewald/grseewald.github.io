\documentclass[11pt,a4paper]{article}

\usepackage[utf8]{inputenc}
\usepackage[ngerman]{babel}

\usepackage{amsmath}
\usepackage{amssymb}
\usepackage{mathtools}
\usepackage{xfrac}

\usepackage{color}
\usepackage[table]{xcolor}
\definecolor{blue}{RGB}{0,0,129}

\usepackage{enumerate}

\usepackage{fancyhdr}

\usepackage{graphicx}

\usepackage{listings}

\usepackage{tikz}

\pagestyle{fancy}

\fancyhf[HR]{Lukas Vormwald, Noah Mehling, Gregor Seewald}
\fancyhf[HL]{Übung:Dienstag 14:00}

\title{Rechenanlagen\\Übungsblatt 7}
\author{Lukas Vormwald \and Noah Mehling \and Gregor Seewald}
\date{Übung :Dienstag 14:00}

\newcommand{\nor}[0]{\ensuremath{\text{ nor }}}
\newcommand{\Aufgabe}[1]{\section{Aufgabe #1}}

\DeclarePairedDelimiter\ceil{\lceil}{\rceil}
\DeclarePairedDelimiter\floor{\lfloor}{\rfloor}

\DeclarePairedDelimiter\abs{\lvert}{\rvert}%
\DeclarePairedDelimiter\norm{\lVert}{\rVert}%

% Swap the definition of \abs* and \norm*, so that \abs
% and \norm resizes the size of the brackets, and the
% starred version does not.
\makeatletter
\let\oldabs\abs
\def\abs{\@ifstar{\oldabs}{\oldabs*}}
%
\let\oldnorm\norm
\def\norm{\@ifstar{\oldnorm}{\oldnorm*}}
\makeatother

\begin{document}

\maketitle

  \Aufgabe{1}

    \begin{tabular}{|c|c|c||c|}\hline
      B&L&R&K\\ \hline \hline
      0&0&0&0\\ \hline
      0&0&1&1\\ \hline
      0&1&0&1\\ \hline
      0&1&1&0\\ \hline
      1&0&0&0\\ \hline
      1&0&1&$x$\\ \hline
      1&1&0&$x$\\ \hline
      1&1&1&1\\ \hline

    \end{tabular}

     \begin{figure}[bt]
       \centering
       \scalebox{0.6}
       {
       \begin{tikzpicture}[scale=5]
         \tikzstyle{vertex}=[circle,minimum size=20pt,inner sep=0pt]
         \tikzstyle{selected vertex} = [vertex, fill=red!24]
         \tikzstyle{selected edge} = [draw,line width=5pt,-,red!50]
         \tikzstyle{edge} = [draw,thick,-,black]
         \node[vertex] (v0) at (0,0) {$000$};
         \node[vertex] (v1) at (0,1) {$001$};
         \node[vertex] (v2) at (1,0) {$001$};
         \node[vertex] (v3) at (1,1) {$011$};
         \node[vertex] (v4) at (0.23, 0.4) {$100$};
         \node[vertex] (v5) at (0.23,1.4) {$101$};
         \node[vertex] (v6) at (1.23,0.4) {$011$};
         \node[vertex] (v7) at (1.23,1.4) {$111$};
         \draw[edge] (v0) -- (v1) -- (v3) -- (v2) -- (v0);
 	       \draw[edge] (v0) -- (v4) -- (v5) -- (v1) -- (v0);
 	       \draw[edge] (v2) -- (v6) -- (v7) -- (v3) -- (v2);
 	       \draw[edge] (v4) -- (v6) -- (v7) -- (v5) -- (v4);
       \end{tikzpicture}
       }
     \end{figure}

\end{document}
