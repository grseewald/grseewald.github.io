\documentclass[11pt,a4paper]{article}

\usepackage[utf8]{inputenc}
\usepackage[ngerman]{babel}
\usepackage{amsmath}
\usepackage{enumerate}
\usepackage{fancyhdr}
\usepackage{xfrac}
\usepackage{amssymb}
\usepackage{mathtools}
\pagestyle{fancy}
\fancyhf[HR]{Lukas Vormwald, Gregor Seewald}
\fancyhf[HL]{Übung: Dienstag 14:00}

\title{Rechenanlagen - Übungsblatt 2}
\author{Lukas Vormwald \and Noah Mehling \and Gregor Seewald}
\date{Übung: Dienstag 14:00}
\begin{document}
	\maketitle
	\section*{Aufgabe 2.1}
		\begin{enumerate}[a)]
			\item $R^i$ ist die Menge aller Knoten, die in einem Radius von $i-Verbindungen$ vom Startknoten aus liegen.\\
			$R^*$ ist die Menge aller Knoten-Tupel die vom Startknoten erreichbar sind.
			\item $n=3$, da wenn man von Knoten $7,8,9$ startet, nur drei Tupel möglich sind:$\left\lbrace(7,8),(8,9),(9,7) \right\rbrace$
			\item Diese Relation beschreibt die Tupel, die in der Teilmenge C verbunden sind und die leere Verbindung.\\
			Die Menge $\left\lbrace(7,8),(8,9),(9,7) \right\rbrace$ bildet eine Äquivalenzrelation.\\
			\begin{itemize}
				\item Reflexivität: Da die leere Verbindung eine Verbindung eines Knotens auf sich selbst definiert ist, ist die Reflexivität gegeben.
				\item Transitivität: Aus der Definition von $R^i$ folgt, dass Knoten, die über einen anderen Knoten verbunden sind ebenfalls verbunden sind, somit gilt:$a\rightarrow b, b\rightarrow c, a \Rightarrow c$
				\item Symmetrie: Die angegebene Menge ist ein Ring aus Verbindungen, daher erreicht man über Umwege immer wieder das Ausgangselement.
			\end{itemize}
		\end{enumerate}
	\section*{Aufgabe 2.2}
		\begin{enumerate}[a)]
			\item wähle $a=01$ und $b=11$. Ohne führende Nullen gelesen ist der Text $"11"$ sowohl als $"aa"$ als auch als $"b"$ interpretierbar und somit nicht eindeutig, also auch keine Codierung.
			\item Da $0^x$ immer $0$ ist (für $x>0$) ist diese Abbildung immer eine Abbildung auf $0$ für $a_0 ... a_{n-2}$ und somit ebenfalls nicht eindeutig definiert. 
		\end{enumerate}\newpage
	\section*{Aufgabe 2.3}
		\begin{tabular}{|c|c|c|c|c|c|c|}
			\hline
			&kein Verband&nicht distributiv&nicht komplementär&\begin{tabular}{c}
			weder distributiv\\
			noch komplementär
			\end{tabular}&inf&sup\\
			\hline\hline
			1.&X&$\surd$&X&X&a&e\\
			\hline
			2.&X&&$\surd$&&a&e\\
			\hline
			3.&$\surd$&&&&&\\
			\hline
		\end{tabular}
\end{document}