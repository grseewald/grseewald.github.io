\documentclass[11pt,a4paper]{article}

\usepackage[utf8]{inputenc}
\usepackage[ngerman]{babel}
\usepackage{amsmath}
\usepackage{enumerate}
\usepackage{fancyhdr}
\usepackage{xfrac}
\usepackage{amssymb}
\usepackage{mathtools}
\pagestyle{fancy}
\fancyhf[HR]{Lukas Vormwald, Gregor Seewald}
\fancyhf[HL]{Übung: Dienstag 14:00}

\title{Rechenanlagen - Übungsblatt 1}
\author{Lukas Vormwald \and Gregor Seewald}
\date{Übung: Dienstag 14:00}
\begin{document}
\maketitle
	\section*{Aufgabe 1.1}
		\begin{enumerate}
			\item $n\cdot m$
			\item $\left( n \cdot m \right)!$
		\end{enumerate}
	\section*{Aufgabe 1.2}
	\begin{tabular}{|c|c|c|c|c|}
	\hline 
	 & Injektiv & Surjektiv & Total & Bijektiv \\ 
	\hline 
	1 & $\surd$ & X & X & X \\ 
	\hline 
	2 & X & X & X & X \\
	\hline
	3 & X & X & X & X \\
	\hline  
	\end{tabular} 
	\section*{Aufgabe 1.3}
	\begin{enumerate}
		\item
		\item 	$b=\left\lbrace \right\rbrace \qquad a=lol$\\
				$\rotatebox[origin=c]{180}{$\Lsh$}  \left\lbrace\right\rbrace l\left\lbrace\right\rbrace ol\left\lbrace\right\rbrace$\\
				$b \neq a$\\
				Wenn b=a, dann kann b als prä-, in- und suffix von a mit der leeren Menge als Nach/Vorfolger ausgedrückt werden. 
	\end{enumerate}
\end{document}