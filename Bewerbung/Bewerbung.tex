%% start of file `template.tex'.
%% Copyright 2006-2013 Xavier Danaux (xdanaux@gmail.com).
%
% This work may be distributed and/or modified under the
% conditions of the LaTeX Project Public License version 1.3c,
% available at http://www.latex-project.org/lppl/.


\documentclass[11pt,a4paper,sans]{moderncv}        % possible options include font size ('10pt', '11pt' and '12pt'), paper size ('a4paper', 'letterpaper', 'a5paper', 'legalpaper', 'executivepaper' and 'landscape') and font family ('sans' and 'roman')

% moderncv themes
\moderncvstyle{banking}                             % style options are 'casual' (default), 'classic', 'oldstyle' and 'banking'
\moderncvcolor{blue}                               % color options 'blue' (default), 'orange', 'green', 'red', 'purple', 'grey' and 'black'
%\renewcommand{\familydefault}{\sfdefault}         % to set the default font; use '\sfdefault' for the default sans serif font, '\rmdefault' for the default roman one, or any tex font name
%\nopagenumbers{}                                  % uncomment to suppress automatic page numbering for CVs longer than one page

% character encoding
\usepackage[utf8]{inputenc}                       % if you are not using xelatex ou lualatex, replace by the encoding you are using
\usepackage[ngerman]{babel}
%\usepackage{CJKutf8}                              % if you need to use CJK to typeset your resume in Chinese, Japanese or Korean

% adjust the page margins
\usepackage[scale=0.75]{geometry}
%\setlength{\hintscolumnwidth}{3cm}                % if you want to change the width of the column with the dates
%\setlength{\makecvtitlenamewidth}{10cm}           % for the 'classic' style, if you want to force the width allocated to your name and avoid line breaks. be careful though, the length is normally calculated to avoid any overlap with your personal info; use this at your own typographical risks...

% personal data
\name{Gregor}{Seewald}
%\title{}                               % optional, remove / comment the line if not wanted
\address{Alte Poststraße 31}{97297 Waldbüttelbrunn}{Deutschland \medskip}% optional, remove / comment the line if not wanted; the "postcode city" and "country" arguments can be omitted or provided empty
\phone[mobile]{+49~(176)~41623908}                   % optional, remove / comment the line if not wanted; the optional "type" of the phone can be "mobile" (default), "fixed" or "fax"
\phone[fixed]{(0931)~408357}
%\phone[fax]{+3~(456)~789~012}
\email{grseewald@gmail.com}                               % optional, remove / comment the line if not wanted
% \homepage{www.johndoe.com}                         % optional, remove / comment the line if not wanted
% \social[linkedin]{john.doe}                        % optional, remove / comment the line if not wanted
% \social[twitter]{jdoe}                             % optional, remove / comment the line if not wanted
% \social[github]{jdoe}                              % optional, remove / comment the line if not wanted
% \extrainfo{optional}                 % optional, remove / comment the line if not wanted
%\photo[4,0cm][0.2pt]{bewerbungsbild.png}                       % optional, remove / comment the line if not wanted; '64pt' is the height the picture must be resized to, 0.4pt is the thickness of the frame around it (put it to 0pt for no frame) and 'picture' is the name of the picture file
% \quote{Some quote}                                 % optional, remove / comment the line if not wanted

% to show numerical labels in the bibliography (default is to show no labels); only useful if you make citations in your resume
%\makeatletter
%\renewcommand*{\bibliographyitemlabel}{\@biblabel{\arabic{enumiv}}}
%\makeatother
%\renewcommand*{\bibliographyitemlabel}{[\arabic{enumiv}]}% CONSIDER REPLACING THE ABOVE BY THIS

% bibliography with mutiple entries
%\usepackage{multibib}
%\newcites{book,misc}{{Books},{Others}}
%----------------------------------------------------------------------------------
%            content
%----------------------------------------------------------------------------------
\begin{document}

% #################################
% Anschreiben.............

\recipient{Institut für Mathematik}{Emil-Fischer-Straße 40\\
97074 Würzburg\\
Gebäude:  40 (Mathematik Ost)\\
Raum:  00.009\\\vspace{1cm}}

\date{\today}
\opening{Sehr geehrter Herr Dr. Greiner,}
\closing{Mit freundlichen Grüßen, %\vspace{0,3cm} \\ %\includegraphics[width=4.5cm]{unterschrift.png} \vspace{-1,0cm}%
}
\enclosure[Anhänge]{Lebenslauf}          % use an optional argument to use a string other than "Enclosure", or redefine \enclname
\makelettertitle

%uns wurde nach der vergangenen Mathematik Präsenzübung eine Stelle als Sysadmin-Hiwi in ihrem Institut vorgestellt. Gerne bewerbe ich mich auf diese Stelle.\\

hiermit möchte ich mich für die Stelle als Sysadmin-Hiwi in ihrem Institut bewerben, welcher uns nach der letzten Mathematik Präsenzübung vorgestellt wurde.

Vorkenntnise habe ich in soweit, dass ich seit mehreren Jahren auf meinen Heimrechnern Linux nutze und auch zwei Server im Einsatz habe.\\
Schon immer habe ich lieber die \glqq bash\grqq genutzt als mich durch Graphische Oberflächen zu hangeln. Seit zwei Monaten habe ich auch \glqq zsh\grqq im Einsatz.\\
Meine Übungsblätter für die verschiedenen Vorlesungen, texe ich schon seit dem ersten Semester, und habe \LaTeX{} auch schon für meine Seminararbeit genutzt.

Über eine Einladung zum Auswahlgespräch würde ich mich sehr freuen.\\


\bigskip


\makeletterclosing

\newpage

% #################################
% Lebenslauf......................

\makecvtitle

\section{Ausbildung}
% arguments 3 to 6 can be left empty
\cventry{09/10--04/18}{Abitur}{Friedrich-Koenig-Gymnasium}{Würzburg}{\textit{Note: "`2,7"'}}{Allgemeine Hochschulreife}
\cventry{seit 10/18}{Studium}{Julius-Maximilians-Universität}{Würzburg}{Studium der Informatik}{}

%\section{Masterarbeit}
%\cvitem{Thema}{\emph{Pressures Produced When Penguins Pooh -- Calculations on Avian Defaecation", Note: 1,0}}
%\cvitem{Betreuer}{Prof. Dr. Tom Tux, Dipl.-Ing. Tina Tinker}
%\cvitem{Abstrakt}{Lorem ipsum dolor sit amet, consectetuer adipiscing elit. Aenean commodo ligula eget dolor. Aenean massa. Cum sociis natoque penatibus et magnis dis parturient montes, nascetur ridiculus mus. Donec quam felis, ultricies nec, pellentesque eu, pretium quis, sem.}

%\section{Bachelorarbeit}
%\cvitem{Thema}{\emph{Untersuchung der Gleitfähigkeit von Bananenschalen, Note: 1,5}}
%\cvitem{Betreuer}{Prof. Dr. Andreas Code, Dipl.-Ing. Johan Github}
%\cvitem{Abstrakt}{Donec pede justo, fringilla vel, aliquet nec, vulputate eget, arcu. In enim justo, rhoncus ut, imperdiet a, venenatis vitae, justo. Nullam dictum felis eu pede mollis pretium. Integer tincidunt. Cras dapibus. Vivamus elementum semper nisi. Aenean vulputate eleifend tellus.}

%\newpage

%\section{Berufserfahrung}
%\cventry{07/03--02/04}{Werkstudent}{Fraunhofer-Institut für Pinguinkunde}{Tuxhausen}{}{Mitarbeit im Labor: Lorem ipsum dolor sit amet, consectetuer adipiscing elit. Aenean commodo ligula eget dolor. Aenean massa. Cum sociis natoque penatibus et magnis dis parturient montes, nascetur ridiculus mus.}
%\cventry{10/00--12/02}{Wissenschaftliche Hilfskraft}{Institut für Angewandtes Nichtstun}{Karl-Tux-Stadt}{}{Mitarbeit in der Verwaltung: Lorem ipsum dolor sit amet, consectetuer adipiscing elit. Aenean commodo ligula eget dolor. Aenean massa. Cum sociis natoque penatibus et magnis dis parturient montes, nascetur ridiculus mus.}

\section{Praktika}
%\cventry{10/01--04/02}{Fachpraktikum}{Tux GmbH}{Debianstadt}{}{Lorem ipsum dolor sit amet, consectetuer adipiscing elit. Aenean commodo ligula eget dolor. Aenean massa. Cum sociis natoque penatibus et magnis dis parturient montes, nascetur ridiculus mus. Donec quam felis, ultricies nec, pellentesque eu, pretium quis, sem. Nulla consequat massa quis enim.}
\cventry{}{Grundpraktikum}{Zentrum für Angewandte Energieforschung}{Würzburg}{}{}

%\section{Soziales Engagement}
%\cventry{10/91--06/92}{Linux-Nachhilfe}{}{Tuxdorf}{}{Donec pede justo, fringilla vel, aliquet nec, vulputate eget, arcu. In enim justo, rhoncus ut, imperdiet a, venenatis vitae, justo. Nullam dictum felis eu pede mollis pretium. Integer tincidunt. Cras dapibus. Vivamus elementum semper nisi.}

\section{Sprachen}
\cvitemwithcomment{Deutsch}{Muttersprache}{}
\cvitemwithcomment{Englisch}{Verhandlungssicher}{}
%\cvitemwithcomment{Spanisch}{Gute Kenntnisse}{}
%\cvitemwithcomment{Portugisisch}{Grundkenntnisse}{}

\section{EDV}
\cvitem{Betriebssysteme}{Linux (Arch, Ubuntu, Debian), Windows}
\cvitem{Programmieren}{Java, C}
\cvitem{Office}{Microsoft Office, LibreOffice}
\cvitem{Textsatz}{\LaTeX}

%\section{Referenzen}
%\cvitem{Prof. Dr.-Ing. habil. Tom Tuxer}{Fraunhofer Institut für Pinguinkunde, tom.tuxer@pingu.edu, (098)~7654~321}

% Publications from a BibTeX file without multibib
%  for numerical labels: \renewcommand{\bibliographyitemlabel}{\@biblabel{\arabic{enumiv}}}% CONSIDER MERGING WITH PREAMBLE PART
%  to redefine the heading string ("Publications"): \renewcommand{\refname}{Articles}
\nocite{*}
\bibliographystyle{plain}
\bibliography{publications}                        % 'publications' is the name of a BibTeX file

% Publications from a BibTeX file using the multibib package
%\section{Publications}
%\nocitebook{book1,book2}
%\bibliographystylebook{plain}
%\bibliographybook{publications}                   % 'publications' is the name of a BibTeX file
%\nocitemisc{misc1,misc2,misc3}
%\bibliographystylemisc{plain}
%\bibliographymisc{publications}                   % 'publications' is the name of a BibTeX file

\clearpage

\end{document}
%% end of file `template.tex'.
